% $date: 2015-06-15
% $timetable:
%   g10r1:
%     2015-06-15:
%       1:
%       2:
%   g10r2:
%     2015-06-15:
%       1:
%       2:

\section*{Олимпиада}

% $authors:
% - Владимир Брагин
% - Иван Митрофанов
% - Леонид Попов

% $groups$delegate: {not: {and: [g10r1, g10r2]}}

% $matter[first-part]:
% - verbatim: |-
%     \begingroup
%     \newcommand{\iffirstpart}{\iftrue}
%     \newcommand{\ifsecondpart}{\iffalse}
% - .[-first-part]
% - verbatim:
%     \endgroup

% $matter[second-part]:
% - verbatim: |-
%     \begingroup
%     \newcommand{\iffirstpart}{\iffalse}
%     \newcommand{\ifsecondpart}{\iftrue}
% - .[-second-part]
% - verbatim:
%     \endgroup

% $build$style[print,first-part]:
% - .[a6paper,landscape,2-on-1]
% - resize font: [9.40, 11.28]

% $build$style[print,second-part]:
% - .[a6paper,landscape,4-on-1]

\begingroup
\providecommand{\iffirstpart}{\iftrue}
\providecommand{\ifsecondpart}{\iftrue}

\ifsecondpart
\newcommand\additionalsection{\subsection*{Добавка}}
\else
\newcommand\additionalsection{}
\fi

\iffirstpart

\begin{problems}

\item
В~некоторой стране суммарная зарплата $10\%$ самых высокооплачиваемых
работников составляет $90\%$ зарплаты всех работников.
Может~ли так быть, что в~каждом из~регионов, на~которые делится эта страна,
зарплата любых $10\%$ работников составляет не~более $11\%$
всей зарплаты, выплачиваемой в~этом регионе?

\item
При каком наименьшем натуральном $n$ число $2015!$ не~делится на~$n^n$?

\item
На~сторонах $AB$ и~$AD$ квадрата $ABCD$ выбраны точки $N$ и~$P$ соответственно,
а~на~отрезке~$AN$ выбрана точка~$Q$ так, что $NP = NC$
и~$\angle QPN = \angle NCB$.
Докажите, что $\angle BCQ = \frac{1}{2} \angle AQP$.

\item
В~конечной последовательности, состоящей из~натуральных чисел, встречается
ровно $2015$ различных чисел.
Известно, что если из~какого-нибудь члена этой последовательности вычесть $1$,
то~в~полученной последовательности тоже будет встречаться не~менее $2015$
различных чисел.
Найдите минимальную возможную сумму членов исходной последовательности.

\item
Натуральные числа $x$ и~$y$ таковы, что $2 x^2 - 1 = y^{15}$.
Докажите, что если $x > 1$, то~$x$ делится на~$5$.

\end{problems}

\else

\setproblem{5}

\additionalsection

\fi

\ifsecondpart

\begin{problems}

\item
Гидры состоят из~голов и~шей (любая шея соединяет ровно две головы).
Одним ударом меча можно снести все шеи, выходящие из~какой-то головы $A$ гидры. Но~при этом из~головы $A$ мгновенно вырастает по~одной шее во~все головы,
с~которыми $A$ не~была соединена.
Геракл побеждает гидру, если ему удастся разрубить ее на~две несвязанные шеями
части.
Найдите наименьшее $N$, при котором Геракл сможет победить любую стошеюю гидру,
нанеся не~более, чем $N$ ударов.

\end{problems}

\fi

