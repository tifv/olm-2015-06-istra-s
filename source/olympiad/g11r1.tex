% $date: 2015-06-22
% $timetable:
%   g11r1:
%     2015-06-22:
%       1:
%       2:

% $caption: Тренировочная олимпиада

\section*{Олимпиада}

% $authors:
% - Алексей Доледенок
% - Андрей Меньщиков

\begingroup % \def\gcd
\renewcommand\gcd{\operatorname{\text{НОК}}}

\begin{problems}

\item
На~координатной плоскости нарисовали графики 100 квадратных трехчленов вида
$y = a x^2 + 2 a^2 x - (2 a + 1)^2$ при $a = 1, 2, \ldots, 100$.
Затем отметили все точки их~пересечения.
Сколько получилось различных точек?

\item
Существуют~ли натуральные числа $a$, $b$, $c$ такие, что
$\gcd(a, b) = \gcd(a + c, b + c)$?

\item
Четыре перпендикуляра, опущенные из~вершин выпуклого пятиугольника
на~противоположные стороны, пересекаются в~одной точке.
Докажите, что пятый такой перпендикуляр тоже проходит через эту точку.

\item
Есть 101 заключенный.
Их~собираются казнить необычным способом.
Для этого в~отдельной комнате ставятся в~ряд 100 шкатулок и~в~них
раскладываются произвольным образом таблички с~номерами от~1 до~100.
Затем одного из~заключенных запускают в~комнату.
Он смотрит содержимое всех шкатулок и~имеет право поменять местами две
таблички.
Оставшимся 100 заключенным присваиваются номера от~1 до~100.
После этого их~последовательно вводят в~комнату.
Каждый имеет право посмотреть содержимое ровно 50 шкатулок.
Если в~этих 50 шкатулках каждый найдет табличку, совпадающую со~своим номером,
то~заключенные спасены.
Если хотя~бы один ошибется, то~всех казнят.
Могут~ли заключенные заранее договориться так, чтобы спастись?
(Тот, кто уже открывал шкатулки, никак не~общается с~остальными;
двигать шкатулки нельзя.)

\end{problems}

\endgroup % \def\gcd

