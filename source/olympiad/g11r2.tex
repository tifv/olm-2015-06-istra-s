% $date: 2015-06-22
% $timetable:
%   g11r2:
%     2015-06-22:
%       1:
%       2:

% $caption: Тренировочная олимпиада

\section*{Олимпиада}

% $authors:
% - Алексей Доледенок
% - Андрей Меньщиков

\begin{problems}

\item
На~координатной плоскости нарисовали графики 100 квадратных трехчленов вида
$y = a x^2 + 2 a^2 x - (2 a + 1)^2$ при $a = 1, 2, \ldots, 100$.
Затем отметили все точки их~пересечения.
Сколько получилось различных точек?

\item
В~ящике лежат $1000$ яблок, веса любых двух из~которых отличаются не~более чем
вдвое.
Петя раскладывает их~по~пакетам по $10$~штук в~каждый.
Пакет называется \emph{хорошим}, если в~нем веса любых двух яблок отличаются
не~более чем на~$10\%$.
Какое наибольшее количество хороших пакетов Петя гарантированно сможет
получить?

\item
Дан равносторонний треугольник $ABC$ и~прямая~$l$, проходящая через его центр.
Точки пересечения этой прямой со~сторонами $AB$ и~$AC$ отразили относительно
середин этих сторон соответственно.
Докажите, что проходящая через эти точки прямая касается вписанной окружности
треугольника $ABC$.

\item
Существует~ли бесконечная последовательность натуральных чисел такая, что в~ней
нет квадратов, кубов, \ldots, $2015$-х степеней натуральных чисел, и~никакая
сумма нескольких подряд идущих членов этой последовательности не~является
квадратом, кубом, \ldots, $2015$-й степенью натурального числа?

\end{problems}

