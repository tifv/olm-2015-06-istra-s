% $date: 24--27 июня 2015
% $timetable:
%   gS: {}

\section*{Знакомство с геометрией плоскости Лобачевского}

% $authors:
% - Фёдор Юрьевич Попеленский

% $style[-announcement]:
% - .[announcement]

Если говорить коротко, то~претензии современников к~геометрии
Н.~И.~Лобачевского состояли в~том, что
<<так не~бывает, потому что так не~может быть в~принципе>>.
Спустя 15 лет после смерти Лобачевского Клейн придумал модель, в~которой
<<именно так и~есть>>.

Мы познакомимся с~одной из~моделей плоскости Лобачевского, узнаем, чем эта
геометрия похожа на~евклидову, а~чем отличается.
Стремиться мы будем к~тому, чтобы доказать теорему о~равносоставленности
многоугольников равных площадей в~геометрии Лобачевского
(заодно повторим, как доказывается эта теорема в~евклидовой геометрии).

