% $date: 24--27 июня 2015
% $timetable:
%   gS: {}

\section*{Изопериметрическая задача}

% $authors:
% - Фёдор Львович Бахарев

% $style[-announcement]:
% - .[announcement]

Почему кошка, когда ей холодно, сворачивается клубочком?
Как хитроумная Дидона, основательница Карфагена, перехитрила царя Иарбанта?
Обо всем об~этом можно будет узнать на~спецкурсе.

Будут рассказаны элементарные доказательства разного рода изопериметрических
теорем.
Изопериметрическая задача~--- это задача о~поиске фигуры наибольшей площади при
фиксированном периметре и, возможно, некоторых дополнительных условиях.
Можно в~качестве альтернативы искать фигуру наименьшего периметра при
фиксированной площади.
Такого сорта задачи известны с~глубокой древности, однако строгое решение они
получили значительно позже.

Для понимания курса потребуются базовые познания в~геометрии и~немного
в~тригонометрии.

