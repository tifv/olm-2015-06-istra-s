% $date: 2015-06-24

% $timetable:
%   gS: {}

\section*{Изопериметрическая задача}

% $authors:
% - Фёдор Бахарев

\begin{problems}

\item
Докажите, что из~всех трапеций с~данным периметром и~данными основаниями
наибольшую площадь имеет равнобедренная.

\item
Докажите, что из~всех 100-угольников, вписанных в~данную окружность,
максимальную площадь имеет правильный.

\item
Докажите, что из~всех 100-угольников, вписанных в~данную окружность,
максимальный периметр имеет правильный.

\item
Докажите, что из~всех четырехугольников с~данным набором сторон наибольшую
площадь имеет вписанный.

\item
Докажите, что из~всех четырехугольников с~данными углами (порядок углов тоже
фиксирован) и~данным периметром наибольшую площадь имеет описанный.

\item
Будем говорить, что две фигуры имеют одинаковую форму, если они подобны.
В~интересующих нас задачах мы обычно рассматриваем фигуры, принадлежащие
некоторому <<классу форм>>
(треугольники, выпуклые фигуры, многоугольники с~четным числом сторон и~т.~д.).
Ответом на~поставленную изопериметрическую задачу является некоторая форма.
Предположим, что доказана такая теорема:
\begin{quote}\em
Среди всех фигур данного периметра, принадлежащих классу форм $C$, наибольшую
площадь имеет фигура с~формой~$F$.
\end{quote}
Докажите, что тогда верна и~такая теорема:
\begin{quote}\em
Среди всех фигур данной площади, принадлежащих классу форм $C$, наибольший
периметр имеет фигура с~формой $F$.
\end{quote}

\item
Постройте сеть Штейнера для шести точек, образующих прямоугольник $1 \times 2$.

\item
Постройте сеть Штейнера для пяти точек в~вершинах правильного пятиугольника.

\item
Вася купил квадратный участок земли со~стороной $1 \, \text{км}$.
Он знает, что под землей, на~глубине $1 \, \text{м}$, через его участок
проходит прямолинейный трубопровод с~нефтью.
Вася хочет выкопать несколько прямолинейных канав глубиной $1 \, \text{м}$ так,
чтобы заведомо найти трубопровод.
При этом сеть канав не~обязана быть связной.
Придумайте конструкцию, при которой суммарная длина канав будет меньше,
чем в~сети Штейнера для соответствующего квадрата.

\item
\subproblem
Найдите треугольник диаметра~1 наибольшей площади.
\\
\subproblem
Найдите четырехугольник диаметра~1 наибольшей площади.
Единственен~ли ответ?
\\
\subproblem
Докажите, что есть некоторый 2015-угольник диаметра~1 имеет наибольшую
площадь, то~у~него есть перпендикулярные диагонали.
\\
\subproblem
Докажите, что правильный пятиугольник диаметра~1 имеет наибольшую площадь
среди всех пятиугольников диаметра~1.
\\
\subproblem
Докажите, что правильный 2015-угольник диаметра~1 имеет наибольшую площадь
среди всех 2015-угольников диаметра~1.

\item
Докажите, что из~всех выпуклых фигур диаметра~1 наибольшую площадь имеет круг.

\end{problems}

