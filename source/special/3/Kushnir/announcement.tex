% $date: 24--27 июня 2015
% $timetable:
%   gS: {}

\section*{Пучки окружностей}

% $authors:
% - Андрей Юрьевич Кушнир

% $style[-announcement]:
% - .[announcement]

Некоторые планиметрические задачи решаются с~помощью выхода в~трехмерное
пространство.
На~спецкурсе мы будем заниматься обратным: стереометрические задачи засовывать
в~плоскость.
Будет построено некоторое новое геометрическое преобразование, задающее
двойственность.
К~примеру, задача~1 окажется простым следствием задачи~2.
Предполагается, что спецкурс будет сложным, но~интересным.
Для понимания спецкурса необходимо тесное знакомство с~инверсией и~с~аффинными
преобразованиями (а~если вы дружите с~полярными преобразованиями
и~с~проективной геометрией, то~вообще замечательно).
Приглашаются ценители геометрии.

\begingroup
\ifx\problemfigurewidth\undefined
\newlength\problemfigurewidth
\newlength\problemtextwidth
\newlength\problemspacewidth
\fi
\setlength\problemfigurewidth{3cm}
\setlength\problemspacewidth{1em}
\setlength\problemtextwidth{\linewidth}
\addtolength\problemtextwidth{-\problemfigurewidth}
\addtolength\problemtextwidth{-\problemspacewidth}
\begin{minipage}[b]{\problemtextwidth}
\claim{Задача 1}
Сфера касается боковых граней четырехугольной пирамиды и~пересекает
ее~основание.
Докажите, что вторые касательные плоскости к~сфере, проведенные через ребра
основания пирамиды, пересекаются в~одной точке.
\par\vspace{1ex}
\claim{Задача 2}
Смотрите картинку.
\end{minipage}\hspace{\problemspacewidth}%
\begin{minipage}[b]{\problemfigurewidth}
    \jeolmfigure[width=\linewidth]{circles}
\end{minipage}
\endgroup

