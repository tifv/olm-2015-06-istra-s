% $date: 15--18 июня 2015
% $timetable:
%   g9s: {}

\section*{Геометрия тканей}

% $authors:
% - Фёдор Константинович Нилов

% $style[-announcement]:
% - .[announcement]

\begin{minipage}{0.70\textwidth}%
Теория тканей была основана замечательным австрийским геометром Вильгельмом
Бляшке в~20-х годах XX века.
Неформально говоря, тканью называются три множества кривых, которые образуют
триангуляцию некоторой части плоскости (см.~рисунок).
Нас будут интересовать ткани из~простейших кривых~--- прямых и~окружностей.
Оказывается, что задача описания таких тканей на~данный момент остается
нерешенной.
Вместе с~вами мы~будем изучать ткани из~прямых и~окружностей, а~также
рассмотрим красивые геометрические теоремы, которые сильно взаимосвязаны
с~соответствующими тканями.
В~финальной части спецкурса мы~построим недавно открытые примеры тканей
на~плоскости и~посмотрим на~завораживающие картинки тканей на~поверхностях
в~трехмерном пространстве, которые активно применяются в~современной
архитектуре.
\end{minipage}%
\hspace{0.05\textwidth}%
\begin{minipage}{0.25\textwidth}
\jeolmfigure[width=\linewidth]{pappa}
\\[1em]
\jeolmfigure[width=\linewidth]{blaschke}
\end{minipage}

