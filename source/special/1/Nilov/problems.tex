% $date: 15--18 июня 2015
% $timetable:
%   gS: {}

\section*{Задачи к спецкурсу <<Геометрия тканей>>}

% $authors:
% - Фёдор Константинович Нилов

\begin{problems}
\def\R{\ensuremath{R}}
\def\G{\ensuremath{G}}
\def\B{\ensuremath{B}}

\item
Через точку~$A_1$ на~стороне~$AB$ треугольника $ABC$ провели прямую,
перпендикулярную биссектрисе угла~$A$.
Она пересекла сторону~$AC$ в~точке~$A_2$.
Через точку~$A_2$ провели прямую, перпендикулярную биссектрисе угла~$C$.
Она пересекла сторону~$CB$ в~точке~$A_3$.
Аналогично построили точки $A_4$, $A_5$, $A_6$, $A_7$.
Докажите, что $A_7 = A_1$.

\item
Три прямые $l_1$, $l_2$, $l_3$ пересекаются в~одной точке.
Точка~$A_1$ выбирается произвольно.
Точки $A_2$, $A_3$, $A_4$, $A_5$, $A_6$, $A_7$
получаются последовательным отражением точки $A_1$ относительно
прямых $l_1$, $l_2$, $l_3$, а~затем снова $l_1$, $l_2$, $l_3$.
Докажите, что $A_7 = A_1$.

\item
Три луча $l_1$, $l_2$, $l_3$ исходят из~одной точки~$O$.
На~лучах $l_1$ и~$l_2$ выбираются произвольные точки $A_1$ и~$A_2$
соответственно.
На~луче~$l_3$ выбирается такая точка~$A_3$, что угол между $A_2 A_3$ и~$l_3$
равен углу между $A_1 A_2$ и~$l_1$.
Затем на~луче~$l_1$ выбирается такая точка~$A_4$, что угол между $A_3 A_4$
и~$l_1$ равен углу между $A_2 A_3$ и~$l_2$.
Аналогично стоятся точки $A_5$, $A_6$, $A_7$.
Докажите, что $A_7 = A_1$.

\item
На~плоскости отмечены красная ({\R}), зеленая ({\G}) и~синяя ({\B}) точки.
Любая окружность, проходящая ровно через две из~этих точек, окрашена в~цвет
третьей.
Возьмем произвольную точку~$O$ внутри треугольника $RGB$.
Проведем через нее красную, синюю и~зеленую окружность.
На~красной окружности внутри треугольника $RGB$ возьмем произвольную точку
$A_1$.
Проведем через нее зеленую окружность.
Пусть она пересекла синюю окружность через точку $O$ в~точке $A_2$,
отличной от~{\R}, {\G} и~{\B}.
Через точку $A_2$ уже проведены зеленая и~синяя окружность;
проведем красную.
Точку пересечения полученной красной окружности с~зеленой окружностью через
точку $O$, отличную от~{\R}, {\G} и~{\B}, обозначим $A_3$.
Продолжая данное построение, получим точки $A_4$, $A_5$, $A_6$, $A_7$.
Докажите, что $A_7 = A_1$.

\item
Докажите, что три пучка прямых образуют ткань.

\item
Докажите, что следующие множества прямых и~окружностей образуют ткань:
\\
({\R}) прямые, касающиеся единичной полуокружности;
\\
({\G}) касающиеся дополнительной к~ней полуокружности;
\\
({\B}) окружности с~центром в~начале координат.

\item
Докажите, что следующие множества прямых и~окружностей образуют ткань:
\\
({\R}) прямые, касающиеся единичной полуокружности;
\\
({\G}) проходящие через начало координат;
\\
({\B}) окружности с~центром в~начале координат.

\item
Докажите, что следующие множества прямых и~окружностей образуют ткань:
\\
({\R}) прямые, проходящие через начало координат;
\\
({\G}) окружности с~центром в~начале координат;
\\
({\B}) окружности, одновременно касающиеся отрезков
\[
    \{(x,y) \colon x = 0, 0 \leq y \leq 1\}
\quad\text{и}\quad
    \{(x,y) \colon y = 0, 0 \leq x \leq 1\}
\,.\]

\item
Докажите, что следующие множества прямых и~окружностей образуют ткань:
\\
({\R}) прямые, проходящие через точку~$A$;
\\
({\G}) прямые, проходящие через точку~$C$;
\\
({\B}) пучок окружностей с~предельными точками в~точках $A$ и~$C$.

\item
Докажите, что пучок окружностей, проходящих через фокусы коники, и~касательные
прямые к~этой конике образуют ткань.

\item
Докажите, что пучок окружностей с~предельными точками в~фокусах коники
и~касательные прямые к~этой конике образуют ткань.

\item
Докажите, что пучок прямых с~вершиной в~фокусе коники, семейство касательных
прямых к~этой конике и~семейство окружностей, дважды касающихся этой коники,
центры которых лежат на~малой оси, образуют ткань.

\end{problems}

