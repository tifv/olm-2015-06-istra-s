% $date: 15--18 июня 2015
% $timetable:
%   gS: {}

\section*{Задачи к спецкурсу <<Суммы множеств>>}

% $authors:
% - Шкредов Илья Дмитриевич


\subsection*{Задачи на 1 балл}

\begin{problems}

\item
Пусть $A \subseteq \mathbb{Z}$~--- произвольное множество.
Пусть для некоторых $b$ выполнено
\(
    |A \cap (A + b)| = |A| - m
\).
Какое максимальное количество таких $b$ существует для $m = 1$, $2$ и~$3$?

\item
\subproblem
Приведите пример множества, не~являющегося суммой двух других множеств,
у~которых мощности не~менее, чем 2.
\\
\subproblem
Приведите пример множества, не~являющегося суммой двух других множеств,
мощности не~менее, чем $k>2$.

\item
Придумайте какое-нибудь необходимое условие для того, чтобы множество имело
вид $A + A$.
Иными словами: чем суммы $A + A$ отличаются от~произвольных множеств?

\item
Приведите пример конечного множества $A \subset \mathbb{Z}$ такого, что
$|A + A| > |A - A|$.
Наоборот?

\item
Пусть $A, B \subset \mathbb{Z}$, $|A|, |B| > 2$.
Докажите, что $|A + B| = |A| + |B| - 1$ тогда и~только тогда, когда $A$ и~$B$
являются арифметическими прогрессиями с~одинаковыми разностями.

\item
Докажите, что для всякого натурального $k$ выполнено
\[
    |A - 2^{k} \cdot A|
\leq
    |A| \cdot
    \left(
        \frac{|A - 2 \cdot A|}{|A|}
    \right)^k
\;.\]

\item
Пусть $A \subseteq \{ 1,2, \ldots, 100 \}$~--- произвольное множество.
Докажите, что $A$ содержит арифметическую прогрессию длины три, если
\\
\subproblem $|A| = 64$;
\quad
\subproblem $|A| = 51$.

\item
Пусть $A \subseteq \{ 1, 2, \ldots, N \}$~--- произвольное множество без
нетривиальных аддитивных четверок, то~есть без решений уравнения
$a + b = c + d$, где $a, b, c, d \in A$~--- различные.
Докажите, что $|A| < 3 \sqrt{N}$ (то~есть множество~$A$ не~может быть слишком
большим).

\item
Верно~ли, что произвольное множество $A \subseteq \mathbb{N}$, имеющее
положительную верхнюю плотность, содержит бесконечную арифметическую
прогрессию?

\item
Верно~ли, что для произвольного множества $A \subseteq \mathbb{N}$, имеющего
положительную верхнюю плотность, множество $A + A$~--- синдетическое?

\item
Докажите, что в~любую разность можно вписать сумму, и наоборот.
Точнее, пусть $A \subseteq \{ 1, 2, \ldots, N \}$, $|A| = \delta N$.
Тогда $A - A$ содержит $B + B$ и~$|B| \geq 2^{-1} \delta^2 N$.

\item
Пусть $A \subseteq \{ 1, 2, \ldots, N \}$ и~$|A| \geq \delta N$.
Докажите, что $A \pm A$ содержит арифметическую прогрессию длины
$\delta^{-1} \log |A|$.

\item
Докажите, что число решений уравнения $x + y + z = 0$ для $x, y, z \in A - A$
не~меньше $|A - A| \cdot |A|$.

\item
Пусть $A \subseteq \mathbb{R}^{+}$.
Докажите, что $|(A + A) / (A + A)| \geq 2 |A|^2 - 1$.

\end{problems}


\subsection*{Некоторые задачи, которые будут решены в классе}
\resetproblem

\begin{problems}

\item
Пусть $A \subset \mathbb{Z}$.
Докажите, что $|A + A| = 2 |A| - 1$ тогда и~только тогда, когда $A$ является
арифметической прогрессией.

\item
Пусть $A \subset \mathbb{R}^2$~--- выпуклое множество.
Докажите, что $A + A = 2 \cdot A$, где $2 \cdot A = \{ 2a \colon a \in A \}$.

\item
Докажите, что $|A - A| \leq |A + A|^2 / |A|$.

\item
Пусть $B \subseteq \{ 1, 2, \ldots, N \}$~--- базис порядка два.
Докажите, что от~сложения с~этим базисом множества \emph{расширяются},
то~есть для любого~$A$ выполнено $|A + B| \geq \sqrt{N \cdot | B |}$.

\item
Докажите, что $|A + A| \leq |A - A|^2 / |A|$.

\item
Если множество $E \in \mathbb{Z}_p$ такое, что $E + x \subseteq E$, $x \neq 0$,
то~$E = \mathbb{Z}_p$.

\item
\subproblem
Пусть $T$~--- произвольное множество из~отрезка $I = \{ 1, 2, \ldots, N \}$
и~$|T| = \delta N$, $\delta \in (0; 1)$.
Докажите, что $I \subseteq T - T + S$, где $S$~--- некоторое множество такое,
что $|S| \leq 2 \delta^{-1}$.
Покажите, что данная оценка точна по порядку.
\\
\subproblem
Пусть множество $A \subseteq \mathbb{N}$ имеет положительную верхнюю плотность,
(то~есть
\(
    \liminf_{N \to \infty}
        \bigl| A \cap \{ 1, 2, \ldots, N \} \bigr| / N
>
    0
\)).
Докажите, что множество $A - A$~--- синдетическое, то~есть расстояние между
двумя соседними элементами $A - A$ ограничено некоторой абсолютной величиной.

\item
Для произвольного множества $A \subseteq \mathbb{Z}_p$ положим
\(
    Q[A] = (A - A) / (A - A) \setminus \{ 0 \}
\).
Докажите, что если $|A| > \sqrt{p}$, то~$Q[A] = \mathbb{Z}_p$.

\item
Пусть $A \subseteq \mathbb{N}$.
Докажите, что $|A A + A| \geq |A|^2$.

\item
Пусть $A \subseteq \mathbb{R}^{+}$.
Докажите, что $|A A + A A - A A| \geq |A|^2$.

\end{problems}


\subsection*{Задачи на~2--3 балла и нерешенные задачи}
\resetproblem

\begin{problems}

\item
Пусть $A = \{ a_1 < a_2 < \ldots < a_n \} \subseteq \mathbb{R}$~---
выпуклое множество, то~есть такое, что последовательность
$(a_{i+1} - a_i)$~--- монотонно возрастает.
Верно~ли, что $|A + A| \sim |A - A|$, $n \to \infty$?

\item
Пусть $A = \{ a_1 < a_2 < \ldots < a_n \} \subseteq \mathbb{R}$~---
выпуклое множество, то~есть такое, что последовательность
$(a_{i+1} - a_i)$~--- монотонно возрастает.
Докажите, что для всех $x \neq 0$ выполнено
\(
    \bigl| A \cap (x - A) \bigr| \leq |A|^{2/3}
\).
Справедлива~ли эта оценка для $\bigl| A \cap (A + x) \bigr|$?

\item
Пусть $A = \{ a_1 < a_2 < \ldots < a_n \} \subseteq \mathbb{R}$~--- выпуклое
множество, то~есть такое, что последовательность $(a_{i+1} - a_i)$~---
монотонно возрастает.
Докажите, что для некоторой абсолютной константы $k$ выполнено $|k A| > n^2$.

\itemx{*}
Пусть множество натуральных чисел раскрашено в~конечное число цветов.
Докажите или опровергнуть, что найдутся различные натуральные $x$ и~$y$ такие,
что $x + y$ и~$x y$ одного цвета.

\item
Рассмотрим множество $\{ 1, 4, 9, \ldots, n^2 \}$.
Докажите, что $|A + A + A| \gg n^2$.

\itemx{*}\emph{(Гипотеза Виноградова)}\enspace
Докажите, что минимальный квадратичный вычет/\linebreak[0]невычет
в~$\mathbb{Z}_p$ меньше чем $p^{\epsilon}$ для любого $\epsilon > 0$.

\itemx{*}\emph{(Дискретная проблема Какеи для вычетов)}\enspace
Пусть $A \subseteq \mathbb{Z}^*_p$~--- некоторое множество,
$|A| = (p - 1) / 2$.
Пусть также $|Q| = (p - 1) / 2$~--- произвольное множество.
Верно~ли, что для некоторого $q \in Q$ минимальный элемент множества $q A$
меньше чем $p^{\epsilon}$ для любого $\epsilon > 0$?

\itemx{*}\emph{(Вопрос Алона)}\enspace
Пусть $A \subseteq \mathbb{Z} / p \mathbb{Z}$~--- произвольное множество,
$|A| \geq c \sqrt{p}$, где $c > 0$~--- некоторая маленькая абсолютная константа
(например, $c = 1 / 100$).
Могут~ли все числа из~множества $A - A$ быть квадратичными вычетами/невычетами?
\par
Тот~же вопрос, но~пусть, дополнительно, известно, что все суммы $A - A$~---
различные.

\itemx{*}\emph{(Возвращаемость на~торах)}\enspace
Пусть множество $S \subseteq \mathbb{Z}_p$~--- синдетическое, то~есть
расстояние между его любыми соседними элементами ограничено сверху некоторой
абсолютной константой~$d$.
Докажите, что $S + S$ содержит арифметическую прогрессию длины
$p^{\epsilon(d)}$, где величина $\epsilon(d) > 0$ зависит только от~$d$.

\itemx{*}\emph{(Гипотеза Эрдеша--Турана)}\enspace
Пусть $A = \{ n_1 < n_2 < \ldots \} \subseteq \mathbb{Z}$~--- произвольное
множество такое, что ряд
\(
    \sum_{j=1}^\infty
        1 / n_j
\)~--- расходится.
Докажите, что $A$ содержит арифметическую прогрессию любой длины.

\itemx{*}\emph{(Вопрос Эрдеша---Турана о~базисе)}\enspace
Пусть $A \subseteq \mathbb{N}$~--- асимптотический базис порядка два, то~есть
такое множество, что любое достаточно большое натуральное число принадлежит
$A + A$.
Верно~ли, что для любого натурального~$M$ найдется $x$, которое представляется
в~виде суммы $x = a_1 + a_2$, $a_1, a_2 \in A$ не~менее $M$ способами?

\itemx{*}
Пусть $A \subseteq \mathbb{Z}_p$~--- произвольное множество,
$|A| > p^{\epsilon}$, $\epsilon > 0$.
Докажите, что найдется $l$, зависящее только от~$\epsilon$ такое, что
$(l A)^l = \mathbb{Z}^*_p$.

\itemx{*}
Пусть множество $\mathbb{N}$ раскрашено в~конечное число цветов.
Докажите или опровергните, что найдутся различные натуральные $x$, $y$ и~$z$
одного цвета такие, что $x^2 + y^2 = z^2$.

\item
Сколько существует вычетов в~$\mathbb{Z}_p$ у~которых сосед
(правый/левый/оба)~--- невычет?

\item
Пусть $A \subseteq \{ 1, 2, \ldots, N \}$~--- некоторое множество.
Докажите, что найдется не~более $C \cdot N / |A| \cdot \log N$ трансляций $A$,
которые покроют весь отрезок $\{ 1, 2, \ldots, N \}$
(здесь $C > 0$~--- некоторая константа).

\end{problems}

