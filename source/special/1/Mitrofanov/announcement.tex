% $date: 15--18 июня 2015
% $timetable:
%   g9s: {}

\section*{Прямые, слова и кузнечики}

% $authors:
% - Иван Викторович Митрофанов

% $style[-announcement]:
% - .[announcement]

Рассмотрим на~целочисленной плоскости прямую, не~параллельную осям координат
и~не~проходящую через целые точки.
Если двигаться вдоль этой прямой и~записывать порядок чередования
горизонтальных (1) и~вертикальных (0) линий сетки, то~получится бесконечное
в~обе стороны бинарное слово.
Такие слова называются \emph{механическими}, на~спецкурсе мы~будем изучать
их~и~возникающую при этом красивую математику.

\begin{minipage}{0.70\textwidth}
Один из~фактов (мы~обязательно его докажем): если в~бесконечном слове для
любого $n$ есть ровно $n + 1$ различных подслов длины~$n$, то~это слово может
быть получено как механическое для некоторой прямой.
\end{minipage}%
\hspace{0.05\textwidth}%
\begin{minipage}{0.25\textwidth}%
\jeolmfigure[width=\linewidth]{fibonacci-word-cutting-sequence.asy}
\end{minipage}

