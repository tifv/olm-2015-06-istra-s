% $date: 19--22 июня 2015
% $timetable:
%   gS: {}

\section*{Лемма Шпернера}

% $authors:
% - Александр Игоревич Храбров

% $style[-announcement]:
% - .[announcement]

Для любой непрерывной функции~$f$, заданой на~отрезке $[a; b]$, значения
которой также лежат в~этом отрезке, найдется неподвижная точка
(т.~е. такая точка~$x$, для которой $f(x) = x$).
Это утверждение является несложным следствием теоремы о~промежуточных
значениях.
Аналогичное утверждение для квадрата также справедливо.
Оно было впервые доказано Брауэром чуть более ста лет назад, однако его
доказательство гораздо сложнее.
Позже неожиданно оказалось, что гораздо более простым подходом является
комбинаторный путь доказательства.
А~именно, лемма, установленная Шпернером в~1928 году.

\begin{minipage}{0.25\textwidth}
\jeolmfigure[width=\linewidth]{sperner-fig.pdf}
\end{minipage}%
\hspace{0.05\textwidth}
\begin{minipage}{0.70\textwidth}
\claim{Лемма Шпернера}
Пусть некоторый большой треугольник $ABC$ триангулирован (т.~е. разбит
на~несколько маленьких треугольников, любые два из~которых либо имеют общую
сторону, либо общую вершину, либо не~пересекаются).
Вершины большого треугольника покрасили в~три цвета, а~все вершины маленьких
треугольников также покрасили в~эти цвета: внутренние вершины~--- произвольно,
а~для лежащих на~сторонах $ABC$ вершин цвет противоположной вершины
не~используется (т.~е. вершины, лежащие на~стороне~$AB$ красятся только
в~цвета вершин $A$ и~$B$ и~т.~п.).
Тогда в~триангуляции существует маленький треугольник, все три вершины которого
окрашены в~различные цвета.
\end{minipage}

Помимо теоремы Брауэра с~помощью леммы Шпернера можно доказать и~много других
интересных и~нетривиальных утверждений, например, основную теорему алгебры,
утверждающую, что любой многочлен имеет комплексный корень.

На~первой лекции курса будет рассказано как иметь дело с~непрерывными функциями
и~отображениями, установлено одномерное утверждение про неподвижные точки
и~некоторые другие одномерные версии теорем, которые встретятся дальше.
Остальные лекции будут посвящены двумерным задачам: лемме Шпернера,
ее~модификациям и~разнообразным следствиям.

