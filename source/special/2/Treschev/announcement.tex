% $date: 19--22 июня 2015
% $timetable:
%   g9s: {}

\section*{Наивная и не очень теория множеств}

% $authors:
% - Леонид Попов
% - Виктор Трещёв

% $style[-announcement]:
% - .[announcement]

Множество~--- одно из~центральных понятий в~математике и~используется
практически во~всех ее~областях.
Мы~же будем изучать множества сами по~себе, сравнивать их.
Основным (но~не~единственным) понятием, которое мы будем обсуждать, станет
\emph{мощность} множеств.
Мы освоим некоторую технику работы с~этими мощностяи и~научимся понимать,
в~каких множествах <<больше>> элементов, а~в~каких~--- <<меньше>>.
Также мы прикаснемся к~истории становления одной из~самых молодых теорий
математики и~узнаем о~трудностях, которые при этом возникали.

Спецкурс не~требует специальной подготовки и~рассчитан на~самый широкий круг
слушателей.

