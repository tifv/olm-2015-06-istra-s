% $date: 2015-06-16
% $timetable:
%   g11r2:
%     2015-06-16:
%       2:
%   g11r1:
%     2015-06-16:
%       1:

\section*{Взвешивания-1}

% $authors:
% - Владимир Трушков

\begin{problems}

\item
Среди $29$ разложенных в~ряд монет имеется $3$ фальшивые, причем известно, что
они лежат подряд.
Настоящие монеты имеют стандартный вес, а~фальшивые~--- какой попало, но~легче
настоящей.
За~три взвешивания на~рычажных весах выявите все три фальшивые монеты.

\item
Семь монет расположены по~кругу.
Известно, что какие-то четыре из~них, идущие подряд,~--- фальшивые и~что каждая
фальшивая монета легче настоящей.
Объясните, как найти две фальшивые монеты за~одно взвешивание на~чашечных весах
без гирь.
(Все фальшивые монеты весят одинаково.)

\item
Среди $11$ внешне одинаковых монет $10$ настоящих, весящих по~$20\,\text{г}$,
и~одна фальшивая, весящая $21\,\text{г}$.
Имеются чашечные весы, которые оказываются в~равновесии, если груз на~правой
их~чашке ровно вдвое тяжелее, чем на~левой.
(Если груз на~правой чашке меньше, чем удвоенный груз на~левой, то~перевешивает
левая чашка, если больше, то~правая.)
Как за~три взвешивания на~этих весах найти фальшивую монету?

\item
Антиквар приобрел $99$ одинаковых по~виду старинных монет.
Ему сообщили, что ровно одна из~монет~--- фальшивая~--- легче настоящих
(а~настоящие весят одинаково).
Как, используя чашечные весы без гирь, за~$7$~взвешиваний выявить фальшивую
монету, если антиквар не~разрешает никакую монету взвешивать более двух раз?

\item
Имеется $6$~монет, из~которых две~--- фальшивые, тяжелее настоящих
на~$0{,}1$~грамма.
Есть весы, которые реагируют только на~разность весов на~чашках, не~меньшую
$0{,}2$ грамма.
Как найти обе фальшивые монеты за~четыре взвешивания?

\item
Даны $8$ гирек весом $1, 2, \ldots, 8$ граммов, но~неизвестно, какая из~них
сколько весит.
Барон Мюнхгаузен утверждает, что помнит, какая из~гирек сколько весит,
и~в~доказательство своей правоты готов провести одно взвешивание, в~результате
которого будет однозначно установлен вес хотя~бы одной из~гирь.
Не~обманывает~ли он?

\item
Имеется $40$ монет, три из~которых
фальшивые, которые легче настоящих, но~имеют одинаковые массы.
За~три взвешивания определите $18$ настоящих монет.

\item
Имеется $100$ монет, четыре из~которых
фальшивые, которые легче настоящих, но~имеют одинаковые массы.
За~два взвешивания определите $14$ настоящих монет.

\item
При изготовлении партии из~$N \geq 5$  монет работник по~ошибке изготовил
две монеты из~другого материала (все монеты выглядят одинаково).
Начальник знает, что таких монет ровно две, что они весят одинаково,
но~отличаются по~весу от~остальных.
Работник знает, какие это монеты и~что они легче остальных.
Ему нужно, проведя два взвешивания на~чашечных весах без гирь, убедить
начальника в~том, что фальшивые монеты легче настоящих, и~в~том, какие именно
монеты фальшивые.
Может~ли он~это сделать?

\item
Адвокат знает, что все 10~монет весят одинаково.
Как ему убедить в~этом судью, сделав не~более трех взвешиваний на~чашечных
весах без гирь?

\end{problems}

