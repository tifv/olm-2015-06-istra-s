% $date: 2015-06-17
% $timetable:
%   g11r2:
%     2015-06-17:
%       2:
%   g11r1:
%     2015-06-17:
%       1:

\section*{Взвешивания-2}

% $authors:
% - Владимир Трушков

\begin{problems}

\item
Имеется $101$ монета.
Среди них $100$ настоящих монет и~одна фальшивая, отличающаяся от~них по~весу.
Необходимо выяснить, легче или тяжелее фальшивая монета.
Как это сделать при помощи двух взвешиваний на~чашечных весах без гирь?

\item
Имеется $7$~монет, из~которых две~--- фальшивые, весящие меньше настоящих.
За~три взвешивания определите обе фальшивые монеты.

\item
Имеется 6~монет, среди которых не~более двух фальшивых, которые весят
одинаково и~легче настоящих.
Как за~3~взвешивания определить все фальшивые монеты?

\item
Имеется 5 золотых монет, из~которых одна фальшивая легкая, и~5 серебряных
монет, из~которых одна фальшивая легкая.
За~три взвешивания на~чашечных весах без гирь найдите обе фальшивые монеты.
(Настоящая золотая и~настоящая серебряная весят одинаково.)

\item
Среди $21$ внешне одинаковой монеты есть одна фальшивая, она легче остальных.
Имеются чашечные весы, которые оказываются в~равновесии, если груз на~правой
их~чашке ровно вдвое тяжелее, чем на~левой.
Как за~три взвешивания на~этих весах найти фальшивую монету?

\item
Имеется три кучки по~2 монеты, в~каждой из~которых по~одной настоящей и~одной
фальшивой.
Известно, что все настоящие весят одинаково и~что все фальшивые весят
одинаково, причем фальшивые весят легче.
За~2 взвешивания на~чашечных весах без гирь найдите все фальшивые монеты.

\item
Из~девяти монеток только одна настоящая, а~остальные восемь~--- фальшивые.
Четыре из~фальшивых монет весят одинаково и~легче настоящей, другие четыре тоже
весят одинаково, но~тяжелее настоящей.
Найдите настоящую монету за~шесть взвешиваний на~двухчашечных весах без гирь.

\item
Имеется $13$~монет, из~которых две~--- фальшивые, весящие меньше настоящих.
За~четыре взвешивания определите обе фальшивые монеты.

\end{problems}

