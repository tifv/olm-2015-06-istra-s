% $date: 2015-06-20
% $timetable:
%   g10r2:
%     2015-06-20:
%       2:

\section*{Применение графов в сельском хозяйстве}

% $authors:
% - Владимир Алексеевич Брагин
% - Иван Викторович Митрофанов

\begin{problems}

\item
Назовем раскраску вершин данного дерева~$T$ в~три цвета \emph{правильной,}
если любые две соединенные ребром вершины имеют разный цвет.
Рассмотрим граф, вершины которого есть все правильные раскраски~$T$, и две
раскраски соединены ребром, если они различаются ровно в~одной вершине.
Докажите, что этот граф~--- связный.

\item
Имеется $20$~бусинок десяти цветов, по~две бусинки каждого цвета.
Их как-то разложили в~$10$ коробок, по~$2$ бусинки в~каждую коробку.
\\
\subproblem
Докажите, что можно выбрать по~одной бусинке из~каждой коробки так, что все
выбранные будут разного цвета.
\\
\subproblem
Докажите, что число способов такого выбора есть ненулевая степень двойки.

\item
Даны $9$ чисел $a_{1}$, \ldots, $a_{9}$.
Известно, что среди попарных сумм $a_{i} + a_{j}$ ($i \neq j$) как минимум
$29$~целых.
Докажите, что все числа $2 a_{1}$, \ldots, $2 a_{9}$~--- целые.

\item
Связная клетчатая фигура состоит из~$1001$ клетки.
Какое в~ней может быть максимальное количество белых клеток?

\item
Клетчатая плоскость раскрашена в~10 цветов.
(Каждая клетка окрашена в~один цвет.)
Любые две соседние клетки окрашены в~разные цвета.
Назовем пару цветов \emph{хорошей,} если есть две соседние клетки, окрашенные
в~эти цвета.
Какое наименьшее количество хороших пар?

\item
На~плоскости расположено 50 попарно непересекающихся кругов.
Пару кругов назовем \emph{хорошей,} если можно выбрать по~точке с~каждого круга
так, чтобы отрезок, их соединяющий, не~пересекал~бы других кругов.
Какое наименьшее количество хороших пар?

\item
Дан клетчатый квадрат $20 \times 20$.
$M$ его клеток окрашены в~черный цвет, остальные в~белый.
Если в~какой-то момент $3$ из~$4$-х клеток, центры которых являются вершинами
прямоугольника со~сторонами, параллельными сторонам квадрата, окрашены в~черный
цвет, то~через минуту и~четвертая клетка тоже перекрашивается в~черный.
При каком наименьшем $M$ может так оказаться, что через некоторое время весь
квадрат станет черным?

\item
Прямоугольный дачный кооператив разделен $M$ горизонтальными
и~$N$ вертикальными границами на~$(M + 1) \times (N + 1)$ прямоугольных
участков.
Сотрудник земельного ведомства хочет выяснить площадь кооператива.
Для этого он может только спрашивать у~каких-то владельцев участков, чему
равна площадь их участка.
Какого наименьшего числа вопросов ему хватит?

\item
На~плоскости нарисованы $n$~кругов, которые попарно не~пересекаются
(но~могут касаться).
Точки касания отмечены красным цветом.
Докажите, что отмечено не~более $(3 n - 6)$ красных точек.

\item
Петя поставил на~доску $50 \times 50$ несколько фишек, в~каждую клетку~---
не~больше одной.
Докажите, что у~Васи есть способ поставить на~свободные поля этой~же доски
не~более $99$ новых фишек (возможно, ни~одной) так, чтобы по-прежнему в~каждой
клетке стояло не~больше одной фишки, и~в~каждой строке и~каждом столбце этой
доски оказалось четное количество фишек.

\item
В~прямоугольной таблице расставлены действительные, но~нецелые числа так, что
сумма чисел в~каждом столбце и~в~каждой строке целая.
Докажите, что можно округлить каждое число (то~есть заменить на~одно из~двух
ближайших целых чисел) так, чтобы сумма чисел в~каждой строчке и~каждом столбце
не~изменилась.

\item
Дано натуральное число $n > 1$.
Рассмотрим все такие покраски клеток доски $n \times n$ в~$k$ цветов, что
каждая клетка покрашена ровно в~один цвет и~все $k$~цветов встречаются.
При каком наименьшем $k$ в~любой такой покраске найдутся четыре окрашенных
в~четыре разных цвета клетки, расположенные в~пересечении двух строк и~двух
столбцов?

\item
Можно~ли расставить по~кругу 1024 нуля или единицы так, чтобы любая
последовательность из~нулей и~единиц длины 10 встречалась среди 10 соседних
цифр?

\end{problems}

