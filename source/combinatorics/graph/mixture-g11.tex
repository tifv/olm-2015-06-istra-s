% $date: 2015-06-15
% $timetable:
%   g11r2:
%     2015-06-15:
%       2:
%   g11r1:
%     2015-06-15:
%       1:

\section*{Графский разнобой}

% $authors:
% - Владимир Трушков

\begin{problems}

\item
В~классе $17$ учеников.
Известно, что среди любых трех учеников найдутся хотя~бы два друга.
Докажите, что в~классе есть ученик, у~которого не~менее 8 друзей.

\item
В~коллективе из~$30$~человек любых пятерых можно усадить за~круглый стол таким
образом, что каждый будет сидеть со~своим знакомым.
Докажите, что в~этом коллективе найдется компания из~$10$ человек, где каждый
знаком с~каждым.

\item
В~группе из~$n^2$~человек каждый имеет не~более $n$~знакомых среди остальных.
Докажите, что можно выбрать $n$~человек, никакие двое из~которых не~знакомы
друг с~другом.

\item
В~кружке $20$ учеников.
Среди них есть ученик, имеющий среди кружковцев одного друга;
ученик, имеющий среди кружковцев двух друзей;
\ldots;
ученик, имеющий среди кружковцев $14$~друзей.
Докажите, что найдутся трое кружковцев, любые двое из~которых дружат.

\item
$22$~школьника участвовали в~съезде юных писателей.
После съезда каждый из~них прочитал произведения трех юных писателей,
побывавших на~съезде.
Докажите, что из~делегатов съезда можно составить комиссию из~четырех человек
так, что в~комиссии никто не~читал произведения остальных ее~членов.

\item
В~графе $34$~вершины, степень каждой не~менее~4, и~для каждой вершины есть еще
ровно одна вершина той~же степени.
Докажите, что в~этом графе есть три вершины, попарно соединенные ребрами.

\item
В~графе на~$300$ вершинах степень каждой вершины не~менее $190$.
Докажите, что можно выбрать $25$ треугольников, попарно не~имеющих общих
вершин.

\item
На~конгресс приехало $100$ ученых, каждый из~которых сделал доклад.
В~конце каждый заявил, что ему понравилось ровно $83$~доклада, сделанных его
коллегами.
Докажите, что найдутся четверо, каждому из~которых понравились доклады трех
других.

\item
Некоторые города Графинии соединены дорогами.
Из~каждого города выходит не~более $n$~дорог, но~среди каждых $m$~городов есть
два, соединенные дорогой, не~проходящей через другие города.
Какое наибольшее количество городов может быть в~Графинии?

\item
В~стране $210$ городов и~совсем нет дорог.
Король хочет построить несколько дорог с~односторонним движением так, чтобы для
любых трех городов $A$, $B$, $C$, между которыми есть дороги, ведущие
из~$A$ в~$B$ и~из~$B$ в~$C$, не~было~бы дороги, ведущей из~$A$ в~$C$.
Какое наибольшее число дорог он сможет построить?

\end{problems}

