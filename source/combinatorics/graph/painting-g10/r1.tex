% $date: 2015-06-22
% $timetable:
%   g10r1:
%     2015-06-22:
%       2:

\section*{Графы и раскраски}

% $authors:
% - Владимир Алексеевич Брагин
% - Иван Викторович Митрофанов

\begin{problems}

\item
\subproblem
В~графе степень каждой вершины меньше $k$.
Докажите, что его вершины можно покрасить правильным образом в~$k$ цветов.
\\
\subproblem
В~связном графе $2015$ вершин и~стпень каждой не~превосходит $17$.
Докажите, что его вершины можно правильным образом покрасить в~$17$ цветов.

\item
В~графе нет простого пути длины~$10$.
Докажите, что его вершины можно правильным образом раскрасить в~$10$ цветов.
(\emph{Простым} называется путь, все вершины которого различны.)

\item
В~выпуклом многоугольнике провели нескольно диагоналей, диагонали
не~пересекаются во~внутренних точках.
Докажите, что вершины многоугольника можно раскрасить в~три цвета так, чтобы
никакие две одноцветные вершины не~были соединены отрезком.

\item
На~плоскости отметили несколько точек и~провели несколько отрезков между ними
так, что никакие два отрезка не~пересекаются по~внутренней точке.
Докажите, что можно покрасить точки в~$6$ цветов так, что никакие две точки
одного цвета не~будут соединены отрезком.

\item
Пусть в~графе $v$~вершин и~$r$ ребер, а~его вершины можно правильным образом
покрасить в~$k$ цветов.
Докажите, что $k \geq v^2 / (v^2 - 2 r)$.

\item
Докажите, что из~графа~$G$ можно удалить не~более, чем $1 / n$ часть его ребер
так, чтобы вершины оставшегося графа можно было покрасить правильным образов
в~$n$ цветов.

\item
В~какое наименьшее число цветов можно покрасить ребра полного графа на~$1000$
вершинах так, чтобы граф на~ребрах каждого цвета был~бы несвязным?

\item
Можно~ли ребра полного графа на~$n$ вершинах разбить на тройки, образующие
треуголники, если
\\
\subproblem $n = 9$;
\quad
\subproblem $n = 10$;
\quad
\subproblem $n = 27$?

\item
В~какое минимальное количество цветов надо покрасить диагонали, стороны
и~вершины правильного 2011-угольника, чтобы одновременно выполнялись следующие
условия:
\\
\emph{(1)} отрезки одного цвета не~должны иметь общих вершин;
\\
\emph{(2)} цвет вершины должен отличаться от~цвета исходящих из~нее отрезков?

\item
В~кабинете президента стоят $2004$ телефона, любые два из~которых соединены
проводом одного из~четырех цветов.
Известно, что провода всех четырех цветов присутствуют.
Всегда~ли можно выбрать несколько телефонов так, чтобы среди соединяющих их
проводов встречались провода ровно трех цветов?

\item
Ребра связного графа раскрашены в~$N$ цветов, причем из~каждой вершины выходит
ровно по~одному ребру каждого цвета.
В~графе удалили по~одному ребру всех цветов, кроме одного.
Докажите, что он остался связен.

\item
Дан граф~$T$.
Оказалось, что для любого $k$ количество способов правильно покрасить вершины
графа~$T$ в~$k$~цветов равно количеству способов покрасить в~$k$ цветов дерево
на~$n$ вершинах.
Докажите, что $T$~--- дерево на~$n$ вершинах.

\end{problems}

