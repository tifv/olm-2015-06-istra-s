% $date: 2015-06-23
% $timetable:
%   g11r2:
%     2015-06-23:
%       2:
%   g11r1:
%     2015-06-23:
%       1:

\section*{Лемма Холла. Продолжаем разговор!}

% $authors:
% - Владимир Трушков

% $matter[-contained,no-header]:
% - verbatim: \setproblem{10}
% - .[contained]

\begin{problems}

\item
В~игре <<Киллер>> в~живых осталось $n$~юношей и~$n$~девушек.
Каждая девушка охотится за~$k$~юношами и, возможно, за~несколькими девушками.
Кроме того, за~каждым юношей охотится $k$~девушек.
Докажите, что каждая девушка сможет совершить убийство, причем девушкам нет
нужды убивать девушек.

\item
В~некотором районе, состоящем из~нескольких деревень, число женихов равно числу
невест.
Известно, что в~каждой из~деревень общее число женихов и~невест не~превосходит
половины от~числа всех женихов и~невест всего района.
Докажите, что всех этих молодых людей можно поженить так, что в~каждой паре муж
и~жена будут из~разных деревень.

\item\emph{(лемма Холла с~дефицитом)}
Докажите, что если любые $k$ ($1 \leq k \leq n$) юношей знакомы в~совокупности
не~менее чем с~$(k - d)$ девушками, то~$(n - d)$ юношей могут выбрать себе
невесту из~числа знакомых.

\item\emph{(лемма Холла для арабских стран)}
Среди $n$ юношей и~нескольких девушек некоторые юноши знакомы с~некоторыми
девушками.
Каждый юноша хочет жениться на~$m$ знакомых девушках.
Докажите, что они смогут это сделать тогда и~только тогда, когда для любого
набора из~$k$ юношей количество знакомых им в~совокупности девушек
не~меньше $k m$.

\item
Среди $n$~юношей и~нескольких девушек некоторые юноши знакомы с~некоторыми
девушками.
$k$-й юноша хочет жениться на~$a_k$ знакомых девушках.
Докажите, что они смогут это сделать тогда и~только тогда, когда для любого
набора $\{k_1, \ldots, k_m\}$ юношей количество знакомых им в~совокупности
девушек не~меньше $a_{k_1} + \ldots + a_{k_m}$.

\item
В~квадрате $n \times n$ стоят неотрицательные числа так, что в~каждой строке
и~в~каждом столбце сумма равна~$1$.
Докажите, что в~этот квадрат можно поставить $n$ не~бьющих друг друга ладей
так, чтобы под каждой поставленной ладьей было положительное число.

\item
В~школу <<Хогвартс>> поступило $24$ первокурсника.
Известно, что в~Гриффиндоре учатся храбрые, в~Когтевране~--- умные,
в~Пуффендуе~--- старательные, а~в~Слизерине~--- хитрые.
Каждый первокурсник обладает ровно двумя этими качествами, и~все качества
встречаются одинаковое число раз.
Докажите, что Распределительная Шляпа сможет поровну поделить первокурсников
на~факультеты.

\item
Пусть выполнено условие леммы Холла, и~каждый из~$n$~юношей знаком по~меньшей
мере с~$k$~девушками.
Докажите, что супружеские пары могут быть составлены по~крайней мере
$k!$~способами, если $k \leq n$.

\end{problems}

