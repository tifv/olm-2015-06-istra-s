% $date: 2015-06-21
% $timetable:
%   g11r2:
%     2015-06-21:
%       2:
%   g11r1:
%     2015-06-21:
%       1:

\section*{Лемма Холла}

% $authors:
% - Владимир Трушков

\begin{problems}

\item
В~условиях леммы Холла назовем множество из~$k$ юношей критическим, если
совокупное количество знакомых им девушек в~точности равно $k$.
Докажите, что объединение и~пересечение двух критических множеств~---
критическое множество.

\item
$20$~школьников решали $20$~задач.
Каждый решил ровно две задачи, и~каждую задачу решили ровно два школьника.
Докажите, что можно организовать разбор задач так, что каждый школьник
расскажет одну из~решенных им задач и~все задачи будут рассказаны.

\item
В~каждой строчке и~в~каждом столбце таблицы $8 \times 8$ стоит ровно $3$~фишки.
Докажите, что из~них можно выбрать восемь~--- по~одной в~каждой строке
и~столбце.

\item
Квадратный лист бумаги разбит на~сто многоугольников одинаковой площади с~одной
стороны и~на~сто других той~же площади с~обратной стороны.
Докажите, что этот квадрат можно проткнуть ста иголками так, что каждый
из~двухсот многоугольников будет проткнут по~разу.

\item
Из~шахматной доски вырезали $7$~клеток.
Докажите, что на~оставшиеся клетки можно поставить $8$ не~бьющих друг друга
ладей.

\item
Докажите, что ребра двудольного графа, степень каждой вершины которого
равна~$k$, можно правильно раскрасить в~$k$~цветов
(из~каждой вершины должны выходить ребра всех цветов по~одному разу).

\item
Прямоугольник $m \times n$ $(m \leq n)$ называется
\emph{латинским прямоугольником,} если он заполнен натуральными числами от~$1$
до~$n$ так, что в~каждой строчке и~в~каждом столбце стоят разные числа.
Докажите, что латинский прямоугольник можно дополнить до~латинского квадрата.

\item
Пусть $F = \{ E_1, E_2, \ldots, E_n \}$~--- набор конечных подмножеств
множества~$E$.
Множество $S = \{ s_1, s_2, \ldots, s_n \}$ различных элементов из~$E$ такое,
что $s_i$ принадлежит $E_i$ при любом $i$, назовем
\emph{системой представителей.}
Докажите, что система представителей существует тогда и~только тогда, когда
объединение любых $k$ подмножеств из~набора~$F$ содержит не~менее $k$~элементов
множества~$E$ (при любом натуральном~$k$ от~$1$ до~$n$).

\item
В~графе все вершины степени~$3$.
Докажите, что можно так покрасить ребра в~два цвета, что из~каждой вершины
выходят ребра обоих цветов.

\item
Есть $n$~юношей и~$n$~девушек.
Каждый юноша знает хотя~бы одну девушку.
Докажите, что тогда можно некоторых юношей поженить на~знакомых девушках так,
чтобы женатые юноши не~знали незамужних девушек.

\end{problems}

