% $date: 2015-06-18
% $timetable:
%   g11r2:
%     2015-06-18:
%       2:

\section*{Разнобой (комбинаторика)}

% $authors:
% - Николай Крохмаль

\begin{problems}

\item
В~компании из~семи человек любые шесть могут сесть за~круглый стол так, что
каждые два соседа окажутся знакомыми.
Докажите, что и~всю компанию можно усадить за~круглый стол так, что каждые два
соседа окажутся знакомыми.

\item
Двое по~очереди ставят крестики и~нолики в~клетки доски $2015 \times 2015$.
Начинающий ставит крестики, его соперник~--- нолики.
В~конце подсчитывается, сколько имеется строчек и~столбцов, в~которых крестиков
больше, чем ноликов~--- это очки, набранные первым игроком.
Число строчек и~столбцов, где ноликов больше~--- очки второго.
Тот из~игроков, который наберет больше очков~--- побеждает.
Кто побеждает при правильной игре?

\item
Куб $n \times n \times n$ разбит на~кубики $1 \times 1 \times 1$.
Какое минимальное количество граней $1 \times 1$ необходимо
в~нем убрать, чтобы из~любой его части можно было пробраться наружу?

\item
В~коробке лежит полный набор костей домино.
Два игрока по~очереди выбирают из~коробки по~одной кости и~выкладывают
их~на~стол, прикладывая к~уже выложенной цепочке с~любой из~двух сторон
по~правилам домино.
Проигрывает тот, кто не~может сделать очередной ход.
Кто выигрывает при правильной игре?

\item
В~колоде лежит $52$~карты, по~$13$ каждой масти.
Ваня вынимает из~колоды по~одной карте.
Вынутые карты в~колоду не~возвращаются.
Каждый раз перед тем, как вынуть карту, Ваня загадывает какую-нибудь масть.
Докажите, что если Ваня каждый раз будет загадывать масть, карт которой
в~колоде осталось не~меньше, чем карт любой другой масти, то~загаданная масть
совпадает с~мастью вынутой карты не~менее $13$~раз.

\item
В~стране несколько городов, некоторые пары городов соединены дорогами.
При этом из~каждого города выходит хотя~бы три дороги.
Докажите, что существует циклический маршрут, длина которого не~делится на~$3$.

\item
В~клетках таблицы $n \times n$ записаны некоторые числа.
Разрешается вместо любых двух из~них записать в~обе клетки их~среднее
арифметическое.
Найдите все натуральные~$n$, при которых для любой начальной расстановки чисел
в~таблице такими операциями можно добиться того, чтобы во~всех клетках были
записаны одинаковые числа.

\end{problems}

