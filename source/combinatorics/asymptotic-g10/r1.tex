% $date: 2015-06-25
% $timetable:
%   g10r1:
%     2015-06-25:
%       1:

\section*{Асимптотика}

% $authors:
% - Иван Митрофанов

\begin{problems}

\itemy{0}\emph{(разобрана)}
Из~бесконечной доски вырезали какое-то конечное число клеток так, что
расстояние между центрами любых двух вырезанных клеток не~менее $1\,000\,000$.
Обязательно~ли оставшееся можно разбить на~доминошки?
Сторона клетки равна~$1$.

\item
Из~бесконечой шахматной доски вырезали конечное число клеток так, что шахматный
конь не~может обойти оставшуюся доску, побывав в~каждой клетке по~одному разу.
Обязательно~ли две какие-то вырезанные клетки находятся на~расстоянии,
не~превосходящем $100$?

\item
Плоскость покрыта внутренностями конечного числа углов.
Докажите, что сумма их~градусных мер не~менее $360^{\circ}$.

\item
\subproblem
Во~всех клетках бесконечной доски, кроме тех, обе координаты которых делятся
на~$100$, стоит по~фишке.
Каждую фишку сдвинули на~расстояние, не~большее $10\,000$.
Докажите, что хотя~бы одна клетка пустая.
\\
\subproblem
Можно~ли утверждать то~же самое, если изначально фишки стоят во~всех клетках,
кроме клеток с~координатами вида $(100 k; 0)$?

\item
\subproblem
Докажите, что существует число, большее $1\,000\,000$, которое нельзя
представить в~виде суммы квадрата и~куба.
\\
\subproblem
Докажите, что для любого $n > 1$ существует бесконечное число натуральных
чисел, не~представимых в~виде суммы $n$ неотрицательных чисел вида $a^n$.

\item
Верно~ли, что из~любого числа можно получить квадрат, добавляя к~его десятичной
записи не~более $100\,500$ цифр?
Цифры можно вписывать в~любые места.

\item
Плоскость разбита на~равные параллелограммы так, что их~вершины~--- целые
точки, а~внутри внутри параллелограммов целых точек нет.
Докажите, что площадь каждого параллелограмма равна~$1$.

\item
Докажите, что в~отрезке $[0; 1]$ все точки, в~десятичной записи которых нет
цифры~$8$, можно покрыть отрезками с~суммарной длиной меньше $0{,}001$.

\end{problems}

