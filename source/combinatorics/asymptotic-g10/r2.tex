% $date: 2015-06-25
% $timetable:
%   g10r2:
%     2015-06-25:
%       2:

\section*{Подсчёт на бесконечности, выделение конечного куска}

% $authors:
% - Иван Митрофанов

\begin{problems}

\itemy{0}\emph{(разобрана)}
Во~всех клетках бесконечной доски, кроме тех, обе координаты которых делятся
на~$100$, стоит по~фишке.
Каждую фишку сдвинули на~расстояние, не~большее $10\,000$.
Докажите, что хотя~бы одна клетка пустая.

\item
\sp
Из~бесконечной клетчатой доски вырезали некоторые клетки так, что любые три
вырезанные клетки лежат не~менее, чем в~трех вертикальных столбцах.
Докажите, что оставшуюся доску можно разрезать на~доминошки.
\\
\sp
Верно~ли то~же самое, если известно только то, что в~любом квадрате
$100 \times 100$ лежит не~более одной вырезанной клетки?

\item
Даны $n$ арифметических прогрессий с~разностями $d_1, \ldots, d_n$.
Докажите, что если $d_1 + \ldots + d_n < 1$, то~существует число,
не~принадлежащее ни~одной прогрессии.

\item
Докажите, что существует число, большее $1\,000\,000$, которое нельзя
представить в~виде суммы квадрата и~куба.

\item
Докажите, что существует бесконечное число натуральных чисел, не~представимых
в~виде суммы десяти десятых степеней.

\item
Верно~ли, что из~любого числа можно получить квадрат, добавляя к~его десятичной
записи не~более $100\,500$ цифр?
Цифры можно вписывать в~любые места.

\item
Плоскость разбита на~равные многоугольники так, что внутри каждого
многоугольника ровно одна точка с~целыми координатами (а~на~границах
многоугольников целых точек нет).
Докажите, что площади многоугольников равны~$1$.

\item
Докажите, что в~отрезке $[0; 1]$ все точки, в~десятичной записи которых нет
цифры~$8$, можно покрыть отрезками с~суммарной длиной меньше $0{,}001$.

\end{problems}

