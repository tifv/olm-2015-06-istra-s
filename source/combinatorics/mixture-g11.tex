% $date: 2015-06-24
% $timetable:
%   g11r2:
%     2015-06-24:
%       2:
%   g11r1:
%     2015-06-24:
%       1:

\section*{Комбинаторный разнобой}

% $authors:
% - Владимир Трушков

\begin{problems}

\item
Сколькими спообами можно разделить на~команды по~6~человек для игры в~волейбол
группу из~$24$~человек?

\item
Переплетчик должен переплести 15 различных книг в~красный, зеленый, жёлтый
и~синий переплеты.
Сколькими способами он может это сделать, если в~каждый цвет должна быть
переплетена хотя~бы одна книга?

\item
$96$~шариков пронумерованы $1, 1, 1, 2, 2, 2, \ldots, 32, 32, 32$
и~разбросаны в~$14$ коробок таким образом, что в~каждой коробке шарики
с~разными номерами.
Докажите, что можно выбрать две коробки таким образом, чтобы на~всех шариках
в~этих коробках были~бы написаны разные числа.

\item
$15$ волейбольных команд разыграли турнир по~круговой системе, причем каждая
команда одержала по~7~побед.
Сколько окажется в~турнире таких троек, команды которых во~встречах между собой
одержали по~одной победе?

\item
В~Цветочном городе живет 2010 коротышек.
У~них имеется 1234 монеты по~10~копеек и~неограниченный запас монет
по~5~копеек.
Иногда коротышки меняются монетами: один дает другому монету в~10 копеек
и~получает взамен две монеты по~5~копеек.
Как-то вечером каждый коротышка заявил: <<Сегодня я отдал ровно 10 монет>>.
Докажите, что кто-то из~них ошибся.

\item
Кузнечик сидит в~одной из~вершин равностороннего треугольника.
За~один прыжок он может оказаться в~любой из~соседних вершин.
Сколькими различными способами он может за~$10$~прыжков вернуться в~начальную
вершину?

\item
Сколькими способами $m$~птиц попарно различных видов можно рассадить
по~$n$~клеткам попарно различных цветов, если в~каждой клетке должны сидеть
одна или две птицы?

\item
Улитка должна проползти вдоль линий клетчатой бумаги путь длины~$2 n$, начав
и~кончив свой путь в~данном узле.
Сколько возможных различных маршрутов у~улитки?

\item
В~классе $20$~учеников.
Каждый дружит не~менее, чем с~$10$~другими.
Докажите, что в~этом классе можно выбрать две тройки учеников так, чтобы любой
ученик из~одной тройки дружил с~любым учеником из~другой тройки.

\item
Посчитайте количество последовательностей длины~$2 n$, в~которых по~$n$~раз
встречаются числа~$1$ и~$-1$, и~все частичные суммы неотрицательны.

\end{problems}

