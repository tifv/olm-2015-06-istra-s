% $date: 2015-06-16
% $timetable:
%   g10r2:
%     2015-06-16:
%       1:

\section*{Принцип крайнего}

% $authors:
% - Владимир Алексеевич Брагин
% - Иван Викторович Митрофанов

\begin{problems}

\item
Докажите, что никакой выпуклый многоугольник нельзя разрезать на~100 различных
правильных треугольников.

\item
На~плоскости нарисованы два непересекающихся выпуклых многоугольника.
Докажите, что можно провести прямую так, что многоугольники окажутся в~разных
полуплоскостях.

\item
На~плоскости отмечено $n$~точек.
Докажите, что среди середин всевозможных отрезков с~концами в~этих точках
не~менее $(2 n - 3)$ различных точек.

\item
Докажите, что любой выпуклый многоугольник $\Phi$ содержит два
непересекающихся многоугольника $\Phi_{1}$ и~$\Phi_{2}$, подобных $\Phi$
с~коэффициентом $1 / 2$.

\item
По~кругу записаны действительные числа.
В~промежутках между соседними числами написали их средние арифметические,
а~сами числа стерли.
Получился такой~же набор чисел.
Докажите, что все числа изначально были равны.

\item
Существуют~ли такие простые числа $p_{1}$, $p_{2}$, $p_{2015}$, что
$(p_{1}^2 - 1)$ делится на~$p_{2}$, $(p_{2}^2 - 1)$ делится на~$p_{3}$, \ldots,
$(p_{2015}^2 - 1)$ делится на~$p_{1}$?

\item
Докажите, что ни~при каком $n > 1$ число
$1 + 1 / 2 + 1 / 3 + \ldots + 1 / n$ не~является целым.

\item
В~графе $200$ вершин и~степень каждой вершины не~меньше $100$.
Докажите, что существует замкнутый путь, проходящий по~всем вершинам по~одному
разу.

\item
На~прямой имеется $2 n + 1$ отрезок, причем каждый отрезок пересекается как
минимум с~$k$ другими отрезками.
Докажите, что некоторый отрезок пересекается со~всеми остальными.

\item
На~прямой дано 50~отрезков.
Докажите, что либо некоторые 8~отрезков имеют общую точку, либо найдутся
8~отрезков, никакие два из~которых не~имеют общей точки.

\end{problems}

