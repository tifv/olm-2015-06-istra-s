% $date: 2015-06-16
% $timetable:
%   g10r1:
%     2015-06-16:
%       2:

\section*{Принцип крайнего}

% $authors:
% - Владимир Алексеевич Брагин
% - Иван Викторович Митрофанов

\begin{problems}

\item
Докажите, что никакой выпуклый многоугольник нельзя разрезать на~100 различных
правильных треугольников.

\item
На~плоскости нарисованы два непересекающихся выпуклых многоугольника.
Докажите, что можно провести прямую так, что многоугольники окажутся в~разных
полуплоскостях.

\item
На~плоскости отмечено $n$~точек.
Докажите, что среди середин всевозможных отрезков с~концами в~этих точках
не~менее $(2 n - 3)$ различных точек.

\item
Площадь выпуклого многоугольника~$M$ равна 1.
\\
\subproblem
Докажите, что можно найти треугольник площади~$4$, содержащий $M$.
\\
\subproblem
Докажите, что можно найти треугольник площади $1 / 4$, содержащийся в~$M$.

\item
Имеется $13$~гирь, каждая из~которых весит целое число граммов.
Известно, что любые $12$ из~них можно так разложить на~$2$~чашки весов,
по~$6$~гирь на~каждой, что наступит равновесие.
Докажите, что все гири имеют один и~тот~же вес.

\item
На~прямой имеется $2 n + 1$ отрезок, причем каждый отрезок пересекается как
минимум с~$k$ другими отрезками.
Докажите, что некоторый отрезок пересекается со~всеми остальными.

\item
Докажите, что ни~при каком $n > 1$ число
\(
    1 + \frac{1}{2} + \frac{1}{3} + \ldots + \frac{1}{n}
\)
не~является целым.

\item
В~графе $200$ вершин и~степень каждой вершины не~меньше $100$.
Докажите, что существует замкнутый путь, проходящий по~всем вершинам по~одному
разу.

\item
На~вечеринку пришли 100 человек.
Затем те, у~кого не~было знакомых среди пришедших, ушли.
Затем те, у~кого был ровно 1~знакомый среди оставшихся, тоже ушли.
Затем аналогично поступали те, у~кого было ровно 2, 3, 4, \ldots, 99
знакомых среди оставшихся к~моменту их ухода.
Какое наибольшее число людей могло остаться в~конце?

\end{problems}

