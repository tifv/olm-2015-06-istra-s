% $date: 2015-06-18
% $timetable:
%   g10r2:
%     2015-06-18:
%       2:
%   g10r1:
%     2015-06-18:
%       1:

\section*{Упорядочивание-2}

% $authors:
% - Владимир Алексеевич Брагин
% - Иван Викторович Митрофанов

\begin{problems}

\item
На~плоскости отмечено $600$ точек общего положения.
Докажите, что их можно покрасить в~$200$ цветов так, чтобы никакие два отрезка,
соединяющие точки одного цвета, не~пересекались во~внутренних точках.

\item
Есть $2 n$ натуральных чисел, не~превосходящих $n^2$.
Докажите, что среди их попарных разностей есть хотя~бы $3$ одинаковых.

\item
В~99 ящиках лежат яблоки и~апельсины.
Докажите, что можно так выбрать 50 ящиков, что в~них окажется не~менее половины
всех яблок и~не~менее половины всех апельсинов.

\item
Пусть $f(n)$~--- количество несократимых дробей из~полуинтервала $[0; 1)$
со~знаменателем, не~превосходящим $n$.
Например, $f(3) = 4$ и~$f(4) = 6$.
Докажите для любого $n$ неравенство $f(2 n) \geq 2 f(n)$.

\end{problems}

