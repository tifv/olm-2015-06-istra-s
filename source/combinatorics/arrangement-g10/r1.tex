% $date: 2015-06-17
% $timetable:
%   g10r1:
%     2015-06-17:
%       2:

\section*{Упорядочивание}

% $authors:
% - Владимир Алексеевич Брагин
% - Иван Викторович Митрофанов

\begin{problems}

\item
Солдаты построены в~две шеренги по~$n$ человек, так что каждый солдат из~первой
шеренги не~выше стоящего за~ним солдата из~второй шеренги.
В~шеренгах солдат выстроили по~росту.
Докажите, что после этого каждый солдат из~первой шеренги также будет не~выше
стоящего за~ним солдата из~второй шеренги.

\item
При каком наибольшем $n$ существуют $n$~палочек таких, что из~любых трех можно
сложить тупоугольный треугольник?

\item
Наименьшее из~$n$ различных натуральных чисел равно $a$.
Докажите, что их НОК не~меньше $n a$.

\item
Среди 25~жирафов, каждые два из~которых различного роста, проводится конкурс
<<Кто выше?>>.
За~один раз на~сцену выходят пять жирафов, а~жюри справедливо (согласно росту)
присуждает им места с~первого по~пятое.
Каким образом надо организовать выходы жирафов, чтобы после семи выходов
определить первого, второго и~третьего призеров конкурса?

\item
Даны 11~гирь разного веса (одинаковых нет), каждая весит целое число граммов.
Известно, что как ни~разложить гири (все или часть) на~две чаши, чтобы гирь
на~них было не~поровну, всегда перевесит чаша, на~которой гирь больше.
Докажите, что хотя~бы одна из~гирь весит более 35~граммов.

\item
На~кольцевом треке $2 N$ велосипедистов стартовали одновременно из~одной точки
и~поехали с~постоянными различными скоростями (в~одну сторону).
Если после старта два велосипедиста снова оказываются одновременно в~одной
точке, назовем это встречей.
До~полудня каждые два велосипедиста встретились хотя~бы раз, при этом никакие
три или больше не~встречались одновременно.
Докажите, что до~полудня у~каждого велосипедиста было не~менее $N^2$~встреч.

\item
В~таблице $10 \times 10$ записаны числа от~$1$ до~$100$.
В~каждой строке выбирается третье по~величине число.
Докажите, что сумма этих чисел не~меньше суммы чисел хотя~бы в~одной из~строк.

\item
\subproblem
Имеются $300$ яблок, любые два из~которых различаются по~весу не~более, чем
в~два раза.
Докажите, что их можно разложить в~пакеты по~два яблока так, чтобы любые два
пакета различались по~весу не~более, чем в~полтора раза.
\\
\subproblem
Имеются $600$ яблок, любые два из~которых различаются по~весу не~более, чем
в~три раза.
Докажите, что их можно разложить в~пакеты по~четыре яблока так, чтобы любые два
пакета различались по~весу не~более, чем в~полтора раза.

\end{problems}

