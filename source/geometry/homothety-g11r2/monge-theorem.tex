% $date: 2015-06-24
% $timetable:
%   g11r2:
%     2015-06-24:
%       1:

\section*{Теорема о трех колпаках}

% $authors:
% - Алексей Доледенок

\begin{problems}

\item
Рассмотрим две гомотетии:
первую с~центром~$O_1$ и~коэффициентом~$k_1$, а~вторую с~центром~$O_2$
и~коэффициентом~$k_2$.
\\
\subproblem
Докажите, что если $k_1 \cdot k_2 = 1$, то~композиция гомотетий является
параллельным переносом.
\\
\subproblem
Докажите, что если $k_1 \cdot k_2 \neq 1$, то~композиция гомотетий является
гомотетией, центр которой расположен на~прямой $O_1 O_2$.

\item
Сколько существует гомотетий, которые данный отрезок переводят в~данный?
А~данную окружность~--- в~данную?

\item
\subproblem\claim{Теорема о~трех колпаках}
На~плоскости даны три неравных окружности, ни~одна из~которых не~лежит внутри
другой.
Докажите, что 3~точки пересечения их~общих внешних касательных лежат на~одной
прямой.
\\
\subproblem
Сколько троек точек, лежащих на~одной прямой, можно найти, если
рассмотреть еще точки пересечения внутренних касательных?

\item
Трапеции $ABCD$ и~$APQD$ имеют общее основание~$AD$.
Докажите, что точки пересечения прямых $AB$ и~$CD$, $AP$ и~$DQ$, $CQ$ и~$PB$
лежат на~одной прямой.

\item
Дан треугольник $ABC$.
Рассмотрим окружность, касающуюся внутренним образом описанной окружности
треугольника $ABC$ в~точке~$P_A$, а~также сторон $AB$ и~$AC$.
Аналогично определим точки $P_B$ и~$P_C$.
Докажите, что прямые $A P_A$, $B P_B$ и~$C P_C$ пересекаются в~одной точке.

\item
Окружности $\omega_1$, $\omega_2$ и~$\omega_3$ вписаны в~углы
треугольника $ABC$.
Окружность~$\Omega$ касается их~внешним образом в~точках $A_1$, $B_1$ и~$C_1$
соответственно.
\\
\subproblem
Докажите, что прямые $A A_1$, $B B_1$ и~$C C_1$ пересекаются в~одной точке.
\\
\subproblem
Пусть радиусы $\omega_1$, $\omega_2$ и~$\omega_3$ равны.
Докажите, что центр~$\Omega$ лежит на~прямой, соединяющей центр вписанной
и~описанной окружностей треугольника $ABC$.

\item
Дана трапеция $ABCD$ с~основаниями $AB$ и~$CD$.
Пусть $P$~--- произвольная точка на~прямой~$BC$,
$M$~--- середина~$AB$,
$X$~--- точка пересечения $AB$ и~$PD$,
$Q$~--- точка пересечения $AC$ и~$PM$,
$Y$~--- точка пересечения $AB$ и~$DQ$.
Докажите, что $M$~--- середина $XY$.

\item
В~треугольнике $ABC$ на~стороне~$AB$ отметили точку~$D$.
Пусть $\omega_1$ и~$\Omega_1$, $\omega_2$ и~$\Omega_2$~--- соответственно
вписанные и~вневписанные (касающиеся $AB$) окружности треугольников $ACD$
и~$BCD$.
Докажите, что общие внешние касательные
к~паре окружностей $\omega_1$ и~$\omega_2$
и к паре $\Omega_1$ и~$\Omega_2$, отличные от~прямой~$AB$, пересекаются
на~прямой~$AB$.

\end{problems}

