% $date: 2015-06-25
% $timetable:
%   g11r2:
%     2015-06-25:
%       2:

% $matter[-contained,no-header]:
% - verbatim: \section*{Гомотетия}
% - .[contained]

\subsection*{Добавка}

% $authors:
% - Алексей Доледенок

\begin{problems}

\item
В~треугольнике $ABC$ на~стороне~$BC$ отмечена точка~$A_1$.
Оказалось, что радиусы вписанных окружностей треугольников $A B A_1$
и~$A C A_1$ равны.
Докажите, что и~радиусы вневписанных окружностей, касающихся сторон $B A_1$
и~$C A_1$, равны.

\item
На~плоскости даны две непересекающихся окружности $\omega_1$ и~$\omega_2$
с~центрами $O_1$ и~$O_2$ и~радиусами $2 R$ и~$R$ соответственно.
Найдите ГМТ точек пересечения медиан треугольников, у~которых одна вершина
лежит на~$\omega_1$, а~две другие~--- на~$\omega_2$.

\item
Внутри круга расположен треугольник $ABC$.
Окружность~$\omega_a$ касатеся продолжений сторон $AB$ и~$AC$ за~точку~$A$,
а~также границы круга внутренним образом в~точке~$A'$.
Аналогично определим точки $B'$ и~$C'$.
Докажите, что прямые $AA'$, $BB'$, $CC'$ пересекаются в~одной точке.

\item
Отрезок~$AB$ пересекает две равные окружности и~параллелен их~линии центров,
причем все точки пересечения прямой~$AB$ с~окружностями лежат между $A$ и~$B$.
Через точку~$A$ проводятся касательные к~окружности, ближайшей к~$A$, через
точку~$B$~--- касательные к~окружности, ближайшей к~$B$.
Оказалось, что эти четыре касательные образуют четырехугольник, содержащий
внутри себя обе окружности.
Докажите, что в~этот четырехугольник можно вписать окружность.

\end{problems}

