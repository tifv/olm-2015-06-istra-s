% $date: 2015-06-23
% $timetable:
%   g11r2:
%     2015-06-23:
%       1:

\section*{Гомотетия}

% $authors:
% - Алексей Доледенок

\begingroup
    \def\ov{\overline}%

\definition
\emph{Гомотетией} с~центром в~точке~$O$ и~коэффициентом $k \neq 0$ называется
преобразование плоскости, которое каждую точку~$A$ плоскости переводит
в~точку~$A'$ такую, что $\ov{OA'} = k \cdot \ov{OA}$.

\begin{problems}

\item
Докажите, что при гомотетии прямая переходит в~параллельную ей прямую,
а~окружность~--- в~окружность.

\item
Докажите, что точки, симметричные данной относительно середин сторон
некоторого квадрата, образуют квадрат.

\item
На~прямоугольном столе лежит $n$~кругов радиуса~$R$ так, что никакие два
не~имеют общих внутреннх точек.
Верно~ли, что на~этом столе можно разместить $4 n$~кругов радиуса~$R / 2$ так,
чтобы никакие два не~имели общих внутренних точек?

\item
В~окружности~$\omega$ проведена хорда~$AB$.
Найдите геометрической место точек пересечения медиан треугольников $ABC$, где
$C \in \omega$.

\item
\subproblem
Докажите, что треугольники с~попарно параллельными сторонами гомотетичны.
\\
\subproblem
Докажите, что точка пересечения высот $H$, точка пересечения медиан $M$
и~центр описанной окружности $O$ треугольника $ABC$ лежат на~одной прямой,
причем $HM = 2 MO$.
\\
\subproblem
В~треугольнике $ABC$ точки $I_{a}$, $I_{b}$, $I_{c}$~--- центры вневписанных
окружностей, касающихся сторон $BC$, $AC$, $AB$ соответственно,
$A_1$, $B_1$, $C_1$~--- точки касания вписанной окружности со~сторонами
$BC$, $AC$, $AB$ соответственно.
Докажите, что прямые $I_{a} A_1$, $I_{b} B_1$, $I_{c} C_1$ пересекаются
в~одной точке.
\\
\subproblem
Докажите, что ортоцентр треугольника $A_1 B_1 C_1$ лежит на~прямой, соединяющей
центры вписанной и~описанной окружностей треугольника $ABC$.

\item
Дан угол и~точка~$A$ внутри него.
Постройте окружность, вписанную в~угол и~проходящую через точку~$A$.
Сколько решений имеет задача?

\item\claim{Лемма Архимеда}
В~окружности~$\omega$ проведена хорда~$AB$.
Рассмотрим окружность, касающиеся хорды~$AB$ и~данной дуги~$AB$.
Докажите, что прямая, соединяющая две точки касания, проходит через середину
оставшейся дуги~$AB$.

\item
Даны две концентрические окружности.
Постройте прямую, на~которой эти окружности высекают три равных отрезка.

\item
Докажите, что радиус окружности, целиком расположенной внутри треугольника,
не~больше радиуса вписанной окружности.

\end{problems}

\endgroup % \def\ov

