% $date: 2015-06-26
% $timetable:
%   g11r2:
%     2015-06-26:
%       2:

% $caption: Один лучше двух (геометрия)

\section*{Один лучше двух}

% $authors:
% - Алексей Доледенок

\claim{Теорема Штейнера--Понселе}
Любое построение, выполнимое на~плоскости циркулем и~линейкой, можно выполнить
одной линейкой, если нарисована окружность и~отмечен ее~центр.

\begin{problems}

\item
Есть линейка без делений.
\\
\subproblem
Даны две параллельные прямые и~отрезок, лежащий на~одной из~них.
Разделите его пополам.
\\
\subproblem
Даны две параллельные прямые и~отрезок, лежащий на~одной из~них.
Удвойте этот отрезок.
\\
\subproblem
Даны две параллельные прямые и~отрезок, лежащий на~одной из~них.
Разделите его на~$n$ равных частей.
\\
\subproblem
Разделите сторону квадратного стола пополам.
Линии можно проводить только на~поверхности стола.
\\
\subproblem
Разделите сторону квадратного стола на~$n$ равных частей.
Линии можно проводить только на~поверхности стола.
\\
\subproblem
Дана окружность, ее~диаметр~$AB$ и~точка~$P$ не~на~окружности.
Проведите через точку~$P$ прямую, перпендикулярную $AB$.
\\
\subproblem
Дана окружность, ее~диаметр~$AB$ и~точка~$P$ на~окружности.
Проведите через точку~$P$ прямую, перпендикулярную $AB$.

\item
На~плоскости даны окружность и~ее~центр.
Пользуясь только линейкой
\\
\subproblem
из~любой точки проведите прямую, параллельную данной прямой, и~опустите
на~данную прямую перпендикуляр;
\\
\subproblem
на~данной прямой от~данной точки отложите отрезок, равный данному отрезку;
\\
\subproblem
постройте отрезок длиной $a b / c$, где $a$, $b$, $c$~--- длины данных отрезков;
\\
\subproblem
постройте точки пересечения данной прямой~$l$ с~окружностью, центр которой~---
данная точка~$A$, а~радиус равен длине данного отрезка;
\\
\subproblem
постройте точки пересечения двух окружностей, центры которых~--- данные точки,
а~радиусы~--- данные отрезки.
Осознайте, что из~всего вышедоказанного следует теорема Штейнера--Понселе.

\end{problems}

