% $date: 2015-06-26
% $timetable:
%   g11r1:
%     2015-06-26:
%       1:

% $caption: Один лучше двух (геометрия)

\section*{Один лучше двух}

% $authors:
% - Алексей Доледенок

\begin{flushright}\footnotesize
Пусть не входит сюда тот, кто не знает геометрии.
\emph{Платон}
\end{flushright}

\claim{Теорема Штейнера--Понселе}
Любое построение, выполнимое на~плоскости циркулем и~линейкой, можно выполнить
одной линейкой, если нарисована окружность и~отмечен ее~центр.

\claim{Теорема Мора--Маскерони}
Любое построение, выполнимое на~плоскости циркулем и~линейкой, можно выполнить
одним циркулем.

\begin{problems}

\item
На~плоскости даны окружность и~ее~центр.
Пользуясь только линейкой
\\
\subproblem
из~данной точки, не~лежащей на~окружости, опустить перпендикуляр на~данный
диаметр;
\\
\subproblem
из~данной точки, лежащей на~окружости, опустить перпендикуляр на~данный
диаметр;
\\
\subproblem
через произвольную точку проведите прямую, параллельную данной прямой,
и~опустите на~данную прямую перпендикуляр;
\\
\subproblem
на~данной прямой от~данной точки отложите отрезок, равный данному отрезку;
\\
\subproblem
постройте отрезок длиной $a b / c$, где $a$, $b$, $c$~--- длины данных отрезков;
\\
\subproblem
постройте точки пересечения данной прямой~$l$ с~окружностью, центр которой~---
данная точка~$A$, а~радиус равен длине данного отрезка;
\\
\subproblem
постройте точки пересечения двух окружностей, центры которых~--- данные точки,
а~радиусы~--- данные отрезки.
Осознайте, что из~всего вышедоказанного следует теорема Штейнера--Понселе.

\item
Пользуясь только циркулем постройте
\\
\subproblem
отрезок в~$n$~раз длиннее данного;
\\
\subproblem
образ точки~$A$, лежащей вне окружности, при инверсии относительно данной
окружности с~данным центром;
\\
\subproblem
образ точки~$A$, лежащей внутри окружности, при инверсии относительно данной
окружности с~данным центром;
\\
\subproblem
середину отрезка с~данными концами;
\\
\subproblem
центр данной окружности.
Осознайте, как построить окружность, проходящую через три данные точки;
% Делаем инверсию с центром в точке A.  Тогда точка, симметричная A
% относительно BC, при обратной инверсии переходит в центр.
\\
\subproblem
точки пересечения данной окружности с~прямой, проходящей через две данные
точки;
\\
\subproblem
точку пересечения прямых $AB$ и~$CD$, где $A$, $B$, $C$ и~$D$~--- данные точки.
Осознайте, что из~всего вышедоказанного следует теорема Мора--Маскерони;
\\
\subproblem
Пусть циркулем можно чертить окружности радиуса не~больше~1.
Верно~ли, что любое построение, выполнимое на~плоскости циркулем и~линейкой,
можно выполнить этим циркулем?

\item
Есть \emph{двусторонняя линейка} ширины~$a$.
С~помощью нее можно
\\
(1)\enspace
через две данные точки проводить прямую;
\\
(2)\enspace
проводить прямую, параллельную данной и~удаленную от~нее на~расстояние~$a$;
\\
(3)\enspace
через данные две точки $A$ и~$B$, где $AB \geq a$, проводить параллельные
прямые, расстояние между которыми равно $a$ (таких пар прямых две).
\\
Докажите, что с~помощью двусторонней линейки можно выполнить любое построение,
выполнимое с~помощью циркуля и~линейки.

\end{problems}

