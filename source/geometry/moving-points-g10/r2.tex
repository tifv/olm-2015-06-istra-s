% $date: 2015-06-22
% $timetable:
%   g10r2:
%     2015-06-22:
%       2:

\section*{Передвижение точек}

% $authors:
% - Фёдор Бахарев

\begin{problems}

\item
Две мухи ползут по~сторонам угла с~одинаковой постоянной скоростью.
В~какой-то момент они оказались на~одинаковом расстоянии от~некоторой
точки~$A$.
Докажите, что они либо все время от~нее равноудалены, либо этого больше никогда
не~произойдет.

\item
Точки $A$ и~$B$ движутся с~постоянными равными скоростями по~двум
пересекающимся прямым.
Докажите, что можно указать две точки $C$ и~$D$ такие, что в~каждый момент
времени точки $A$, $B$, $C$ и~$D$ лежат на~одной окружности.

\item
Дан треугольник $ABC$.
Из~точек $A$ и~$C$ одновременно с~равными скоростями стартуют две мухи
в~сторону точки~$B$.
Докажите, что середина отрезка между мухами движется по~прямой, параллельной
биссектрисе угла~$B$.

\item
Две окружности пересекаются в~точках $A$ и~$B$.
Из~точки~$A$ одновременно стартуют два велосипедиста.
Каждый из~них едет по~своей окружности против часовой стрелки.
При этом угловые скорости у~них одинаковые.
Докажите, что прямая, соединяющая их, все время проходит через точку~$B$.

\item
Велосипедисты едут по~двум концентрическим окружностям с~центром~$O$
с~одинаковыми угловыми скоростями против часовой стрелки.
В~начальный момент их~положения $A_1$  и~$B_1$ таковы, что
$O A_1 \perp A_1 B_1$.
Докажите, что если в~какой-то момент они располагаются в~точках $A_2$ и~$B_2$,
то~$A_1 A_2$ делит $B_1 B_2$ пополам.

\item
Две окружности с~центрами $O_1$ и~$O_2$ пересекаются в~точках $A$ и~$B$.
Из~точки~$A$ одновременно стартуют два велосипедиста.
Каждый из~них едет по~своей окружности один по~часовой стрелке, другой~---
против.
Угловые скорости у~них одинаковые.
Четырехугольник $A O_1 C O_2$~--- параллелограмм.
Докажите, что велосипедисты постоянно равноудалены от~точки~$C$.

\item
Две окружности пересекаются в~точках $A$ и~$B$.
Из~точки~$A$ одновременно стартуют два велосипедиста.
Каждый из~них едет по~своей окружности против часовой стрелки.
При этом угловые скорости у~них одинаковые.
Докажите, что существует точка, постоянно равноудаленная от~велосипедистов.

\item
Из~точки пересечения $A$ двух окружностей одновременно с~одинаковыми угловыми
скоростями стартовали два велосипедиста, оба против часовой стрелки.
Докажите, что середина отрезка, их~соединяющего, движется по~окружности.

\end{problems}

