% $date: 2015-06-25
% $timetable:
%   g10r2:
%     2015-06-25:
%       1:
%   g10r1:
%     2015-06-25:
%       2:

% $caption: Разнобой-повторение (геометрия)

\section*{Разнобой-повторение}

% $authors:
% - Фёдор Бахарев

\begin{problems}
\let\ov\overrightarrow

\item
Пусть $O$~--- центр окружности, описанной около равнобедренного треугольника
$ABC$ ($AB = AC$), $D$~--- середина стороны~$AB$, а~$E$~--- точка пересечения
медиан треугольника $ACD$.
Докажите, что $OE \perp CD$.

\item
\subproblem
Для любых четырех точек на~плоскости докажите равенство
\[
    \ov{AB} \cdot \ov{CD} +
    \ov{AC} \cdot \ov{DB} +
    \ov{AD} \cdot \ov{BC}
=
    0
\, . \]
\subproblem
Выведите из~этого, что высоты треугольника пересекаются в~одной точке.

\item
Дан набор из~$n$ векторов, $n > 2$.
Назовем вектор набора длинным, если его длина не~меньше длины суммы остальных
векторов набора.
Докажите, что если каждый вектор набора длинный, то~сумма всех векторов набора
равна нулю.

\item
Четыре перпендикуляра, опущенные из~вершин выпуклого пятиугольника
на~противоположные стороны, пересекаются в~одной точке.
Докажите, что пятый такой перпендикуляр тоже проходит через эту точку.

\item
В~треугольнике $ABC$ угол~$A$ равен $60^\circ$.
На~продолжениях сторон $AB$ и~$AC$ выбраны точки $D$ и~$E$ такие, что
$DB = BC = CE$.
Окружность, описанная около треугольника $ADC$, пересекает отрезок~$DE$
в~точке~$F$.
Докажите, что $AF$~--- биссектриса угла $BAC$.

\item
На~стороне~$AB$ треугольника $ABC$ выбрана точка~$D$.
Окружность, описанная около треугольника $BCD$, пересекает сторону~$AC$
в~точке~$M$, а~окружность, описанная около треугольника $ACD$, пересекает
сторону~$BC$ в~точке~$N$ (точки $M$ и~$N$ отличны от~точки~$C$).
Пусть $O$~--- центр описанной окружности треугольника $CMN$.
Докажите, что прямая~$OD$ перпендикулярна стороне~$AB$.

\item
Найдите сумму
\(
    \sin (9^\circ) + \sin (49^\circ) + \sin (89^\circ) +
    \ldots +
    \sin (329^\circ)
\).

\item
Дано 8 действительных чисел: $a$, $b$, $c$, $d$, $e$, $f$, $g$, $h$.
Докажите, что хотя~бы одно из~шести чисел
$a c + b d$, $a e + b f$, $a g + b h$, $c e + d f$, $c g + d h$, $e g + f h$
неотрицательно.

\item
Дан треугольник $ABC$ и~окружность с~центром~$O$, проходящая через вершины
$A$ и~$C$ и~повторно пересекающая отрезки $AB$ и~$BC$ в~различных точках
$K$ и~$N$ соответственно.
Окружности, описанные около треугольников $ABC$ и~$KBN$, имеют ровно две общие
точки $B$ и~$M$.
Докажите, что угол $OMB$~--- прямой.

\end{problems}

