% $date: 2015-06-27
% $timetable:
%   g11r1:
%     2015-06-27:
%       1:

\section*{Точка Лемуана}

% $authors:
% - Алексей Доледенок

\claim{Напоминание\;1}
Пусть точка~$S$ на~стороне~$BC$ треугольника $ABC$ такова, что прямая~$AS$
симметрична медиане~$AM$ относительно биссектрисы угла~$A$.
Тогда $AS$ называется \emph{симедианой}.

\claim{Напоминание\;2}
Симедианы треугольника пересекаются в~одной точке~$L$, которая называется
\emph{точкой Лемуана}.

\claim{Напоминание\;3}
Пусть касательные к~описанной окружности треугольника $ABC$ в~точках $B$ и~$C$
пересекаются в~точке~$P$.
Тогда $AP$ содержит симедиану треугольника $ABC$.

\definition
Пусть точки $B_1$ и~$C_1$ лежат на~лучах $AC$ и~$AB$ соответственно.
Отрезок~$B_1 C_1$ называется \emph{антипараллельным} отрезку~$BC$, если
$\angle A B_1 C_1 = \angle ABC$.

\begin{problems}

\item
\subproblem
Докажите, что $AS$ делит антипараллельный отрезок пополам тогда и~только тогда,
когда $AS$~--- симедиана.
\\
\subproblem
Докажите, что если симедиана~$AS$ делит пополам отрезок~$B_1 C_1$, где точки
$B_1$ и~$C_1$ лежат на~лучах $AC$ и~$BC$, то~$B_1 C_1$ антипараллелен $BC$.

\item
Докажите, что точка Лемуана в~прямоугольном треугольнике с~прямым углом~$C$
является серединой высоты, опущенной из~вершины~$C$.

\item
\subproblem
Через точку~$X$ внутри треугольника провели три отрезка, антипараллельных
его сторонам.
Докажите, что эти отрезки равны тогда и~только тогда, когда $X$~---
точка Лемуана.
\\
\subproblem
В~треугольнике $ABC$ провели три равных отрезка, антипараллельных его
сторонам.
Докажите, что их~концы лежат на~одной окружности.
\\
\subproblem
Через точку~$L$ провели прямые, параллельные сторонам треугольника.
Докажите, что их~точки пересечения со~сторонами треугольника лежат на~одной
окружности.
\\
\subproblem
Докажите, что центр окружности из~предыдущего пункта является серединой
отрезка~$OL$, где $O$~--- центр описанной окружности треугольника $ABC$.

\item
Пусть $a_1$, $b_1$, $c_1$~--- расстояния от~точки~$L$ до~сторон
треугольника $ABC$.
Докажите, что
\(
    a_1 : a
=
    b_1 : b
=
    c_1 : c
\).

%\item
%Прямые, содержащие медианы треугольника $ABC$, вторично пересекают его
%описанную окружность в~точках $A_1$, $B_1$, $C_1$.
%Прямые, проходящие через $A$, $B$, $C$ и~параллельные противоположным сторонам,
%пересекают ее~в~точках $A_2$, $B_2$, $C_2$.
%Докажите, что прямые $A_1 A_2$, $B_1 B_2$, $C_1 C_2$ пересекаются
%в~одной точке.

\item
\subproblem
Пусть $A_1$, $B_1$, $C_1$~--- середины сторон $BC$, $AC$, $AB$ остроугольного
треугольника $ABC$, $H_{a}$, $H_{b}$, $H_{c}$~--- середины высот, проведенных
к~сторонам $BC$, $AC$, $AB$.
Докажите, что прямые $A_1 H_{a}$, $B_1 H_{b}$, $C_1 H_{c}$ пересекаются
в~одной точке.
\\
\subproblem
Докажите, что эта точка~--- точка Лемуана.

\end{problems}

