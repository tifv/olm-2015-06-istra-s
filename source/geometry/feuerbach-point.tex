% $date: 2015-06-27
% $timetable:
%   g11r2:
%     2015-06-27:
%       2:

\section*{Теорема Фейербаха}

% $authors:
% - Алексей Доледенок

\begin{problems}

\item
\subproblem
Докажите, что точки, симметричные ортоцентру треугольника $ABC$ относительно
сторон и~середин сторон треугольника $ABC$ лежат на~описанной окружности
треугольника $ABC$.
\\
\subproblem\claim{Окружность Эйлера}
Докажите, что середины сторон, основания высот и~середины отрезков, соединяющих
вершины треугольника с~ортоцентром, лежат на~одной окружности, причем центр
этой окружности является серединой отрезка~$HO$, где $H$~--- ортоцентр,
$O$~--- центр описанной окружности.

\end{problems}

\definition
Углом между пересекающимися окружностями называется меньший угол между
касательными в~их~точке пересечения.

\begin{problems}

\item
Даны две перпендикулярные окружности.
Докажите, что касательная в~точке их~пересечения к~первой окружности проходит
через центр второй.

\item
Пусть окружности $\alpha$ и~$\beta$ пересекаются в~точках $M$ и~$N$.
Окружность~$\omega$ с~центром на~прямой~$MN$ перпендикулярна $\alpha$.
Докажите, что $\omega$ перпендикулярна окружности~$\beta$.

\end{problems}

\claim{Лемма Архимеда}
Есть окружность~$\omega$ и~ее~хорда~$AB$.
Окружность~$\alpha$ касается $\omega$ в~точке~$C$, а~хорды~$AB$~---
в~точке~$D$.
Тогда прямая~$CD$ проходит через точку~$M$~--- середину дуги~$AB$,
не~содержащую точку~$C$.

\begin{problems}

\item
\subproblem
Докажите, что в~условиях леммы Архимеда окружность~$\gamma$ с~центром
в~точке~$M$ и~радиусом~$MA$ перпендикулярна окружности~$\alpha$.
\\
\subproblem\claim{Критерий Архимеда}
Пусть окружность~$\alpha$ касатеся хорды~$AB$ окружности~$\omega$ в~точке~$D$,
а~также окружность~$\alpha$ перпендикулярна окружности~$\gamma$.
Докажите, что $\alpha$ касается~$\omega$.

\item
Пусть $ABCD$~--- вписанный четырехугольник.
Пусть точки $A_1$ и~$C_1$~--- основания перпендикуляров из~точек $A$ и~$C$
на~диагональ~$BD$, точки $B_1$ и~$D_1$~--- основания перпендикуляров из~точек
$B$ и~$D$ на~диагональ~$AC$.
Докажите, что четырехугольник $A_1 B_1 C_1 D_1$ вписанный.

\item
\subproblem
В~треугольнике $ABC$ точки $A_1$ и~$C_1$~--- основания
перпендикуляров из~вершин $A$ и~$C$ на~биссектрису угла~$B$.
Докажите, что точки касания вписанной и~вневписанной окружностей со~стороной
$AC$, а~также точки $A_1$ и~$C_1$ лежат на~одной окружности.
\\
\subproblem
Докажите, что центром этой окружности является середина стороны~$AC$.

\item
\subproblem
В~треугольнике $ABC$ точки $A_1$ и~$C_1$~--- основания перпендикуляров
из~вершин $A$ и~$C$ на~биссектрису угла~$B$;
$B B_1$~--- высота, $B'$~--- середина стороны~$AC$.
Докажите, что точки $A_1$, $B_1$, $C_1$, $B'$ лежат на~одной окружности.
\\
\subproblem
Рассмотрим точку~$P$~--- середину дуги~$B_1 B'$ окружности Эйлера
треугольника $ABC$.
Докажите, что точка~$P$ равноудалена от~точек $A_1$ и~$C_1$.
Выведите отсюда, что $P$~--- центр описанной окружности
четырехугольника $A_1 B_1 C_1 B'$.

\item\claim{Теорема Фейербаха}
Используя критерий Архимеда, докажите, что вписанная окружность и~окружность
Эйлера треугольника $ABC$ касаются.

\end{problems}

