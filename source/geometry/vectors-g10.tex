% $date: 2015-06-23
% $timetable:
%   g10r2:
%     2015-06-23:
%       1:
%   g10r1:
%     2015-06-23:
%       2:

\section*{Про стрелочки}

% $authors:
% - Фёдор Бахарев

\begin{problems}
\let\ov\overrightarrow

\item
Докажите, что если векторы $(\ov{a} + \ov{c})$ и~$(\ov{a} - \ov{c})$
перпендикулярны, то~$|\ov{a}| = |\ov{c}|$.

\item
На~прямой лежит одиннадцать точек $M_{1}, \ldots, M_{11}$.
Вне прямой дана точка~$F$.
Можно~ли на~отрезках $F M_{1}, \ldots, FM_{11}$ расставить стрелки
так, чтобы сумма всех полученных векторов была равна~$\ov{0}$?

\item
\subproblem
Пусть $M$, $N$, $P$, $Q$~--- середины сторон $AB$, $BC$, $CD$ и~$DE$
выпуклого пятиугольника $ABCDE$;
$F$~--- середина~$MP$, $G$~--- середина~$NQ$.
Докажите, что отрезок~$FG$ параллелен отрезку~$AE$ и~имеет вчетверо меньшую
длину.
\\
\subproblem
У~выпуклого пятиугольника отметили середины сторон, а~потом стерли все, кроме
отмеченных точек.
Восстановите пятиугольник.

\end{problems}

\begingroup
\ifx\problemfigurewidth\undefined
\newlength\problemfigurewidth
\newlength\problemtextwidth
\newlength\problemspacewidth
\fi
\setlength\problemfigurewidth{2cm}
\setlength\problemspacewidth{1em}
\setlength\problemtextwidth{\linewidth}
\addtolength\problemtextwidth{-\problemfigurewidth}
\addtolength\problemtextwidth{-\problemspacewidth}
\begin{minipage}{\problemtextwidth}
\begin{problems}
\item
Точки, разбивающие каждую из~сторон четырехугольника на~три равные части,
соединены естественным образом.
Докажите, что
\\
\subproblem
каждый из~полученных отрезков также разбивается точками пересечения на~три
равные части;
\\
\subproblem
площадь среднего четыреухгольника в~девять раз меньше площади исходного.
\end{problems}
\end{minipage}\hspace{\problemspacewidth}%
\begin{minipage}{\problemfigurewidth}
    \jeolmfigure[width=\linewidth]{quadrilateral}
\end{minipage}
\endgroup

\begin{problems}

\item
На~плоскости дано 2015 векторов, причем среди них есть не~коллинеарные.
Известно, что сумма любых 2014 векторов коллинеарна с~вектором, не~включенном
в~сумму.
Докажите, что сумма всех векторов равна нулевому вектору.

\item
На~стене висят двое правильно идущих совершенно одинаковых часов.
Одни показывают московское время, другие~--- местное.
Минимальное расстояние между концами их~часовых стрелок равно $m$,
а~максимальное~--- $M$.
Найдите расстояние между центрами этих часов.

\item
Cередины противолежащих сторон шестиугольника соединены отрезками.
Oказалось, что точки попарного пересечения этих отрезков образуют
равносторонний треугольник.
Докажите, что проведенные отрезки равны.

\item
В~магическом квадрате $n \times n$, составленном из~чисел $1, 2, \ldots, n^2$
(в~нем суммы чисел во~всех строках и~столбцах равны), центры любых двух клеток
соединили вектором по~направлению от~большего числа к~меньшему.
Докажите, что сумма всех этих векторов равна нулю.

\end{problems}

