% $date: 2015-06-23
% $timetable:
%   g11r1:
%     2015-06-23:
%       2:

\section*{Подделываем подписи}

% $authors:
% - Алексей Доледенок

% $matter[-preamble-package-guard]:
% - preamble package: hyperref
%   condition: [-print]
% - preamble package: subcaption
% - .[preamble-package-guard]

Рассмотрим граф~$G$ на~плоскости.
Будем считать некоторые его вершины \emph{неподвижными}~--- жестко
закрепленными на~плоскости.
Остальные назовем \emph{подвижными}.
Припишем каждому ребру этого графа его длину~--- фиксированное положительное
число.
Наш граф нарисован на~плоскости так, что неподвижные вершины закреплены, ребра
являются отрезками, а~расстояния на~плоскости между вершинами, соединенными
ребром, равны длине ребра.
Будем называть вершины~$G$ \emph{шарнирами}, а~ребра~--- \emph{стержнями}.
Полученную конструкцию назовем \emph{шарнирным механизмом}.
\emph{Конфигурационным пространством} (для какого-то закрепления неподвижных
вершин) назовем множество всевозможных состояний шарнирного механизма.
\emph{Конфигурационным пространством шарнира~$v$} назовем множество точек,
в~которых может располагаться вершина~$v$.
\emph{Кратностью состояния~$s$} вершины~$v$ назовем количество состояний,
в~которых вершина~$v$ находится в~состоянии~$s$.

\begin{problems}

\item
\subproblem
Рассмотрим шарнирный механизм, состоящий из~подвижных шарниров $A$, $B$, $C$,
неподвижного шарнира~$O$ и~ребер $OA$, $AB$, $BC$, $CO$, длина которых
равна~$a$.
Найдите конфигурационное пространство и~кратность состояний вершины~$B$.
\\
\subproblem\claim{Ромб}
\label{geometry/linkage:problem:rombus}%
Что можно добавить в~механизм так, чтобы конфигурационное пространство
шарнира~$B$ не~поменялось, а~замкнутая ломаная $OABC$ всегда была ромбом
(возможно, вырожденным)?
Найдите кратность состояний вершины~$B$.

\item
\subproblem
Рассмотрим шарнирный механизм, состоящий из~подвижных шарниров $A$, $B$, $C$,
неподвижного шарнира~$O$ и~ребер $OA$, $AB$, $BC$, $CO$,
$|OA| = |BC| = a$, $|OC| = |AB| = b$.
Найдите конфигурационное пространство и~кратность состояний вершины~$B$.
\\
\subproblem\claim{Параллелограмм}
Что можно добавить в~механизм так, чтобы конфигурационное пространство
шарнира~$B$ не~поменялось, а~замкнутая ломаная $OABC$ всегда была
параллелограммом (возможно, вырожденный)?
Найдите кратность состояний вершины~$B$.
\\
\subproblem\claim{Антипараллелограмм}
\label{geometry/linkage:problem:antiparallelogram}%
Что можно добавить в~механизм так, чтобы конфигурационное пространство
шарнира~$B$ не~поменялось, а~замкнутая ломаная $OABC$ всегда была
самопересекающейся?
Найдите кратность состояний вершины~$B$.

\begin{figure}[ht]
\strut\hfill
\begin{subfigure}[b]{0.45\linewidth}
\begin{center}
    \jeolmfigure[width=1.0\textwidth]{poncelet}
    \caption{к~задаче~\ref{geometry/linkage:problem:poncelet}.}
    \label{geometry/linkage:problem:poncelet:fig}
\end{center}
\end{subfigure}%
\hfill
\begin{subfigure}[b]{0.35\linewidth}
\begin{center}
    \jeolmfigure[width=1.0\textwidth]{hart}
    \caption{к~задаче~\ref{geometry/linkage:problem:hart}.}
    \label{geometry/linkage:problem:hart:fig}
\end{center}
\end{subfigure}
\hfill\strut
\caption{инверсоры.}
\end{figure}

\item\claim{Инверсор Понселе}
\label{geometry/linkage:problem:poncelet}%
Рассмотрим шарнирный механизм, изображенный
на~рисунке~\ref{geometry/linkage:problem:poncelet:fig}.
$ABCD$~--- ромб из~задачи~\ref{geometry/linkage:problem:rombus},
точка~$O$ лежит на~луче~$CA$ за~точкой~$A$,
шарниры $O$ и~$R$ закреплены, стержни изображены сплошными линиями.
Найдите конфигурационное пространство точки~$C$.

\item\claim{Инверсор Гарта}
\label{geometry/linkage:problem:hart}%
Рассмотрим шарнирный механизм, изображенный
на~рисунке~\ref{geometry/linkage:problem:hart:fig}.
$ABCD$~--- антипараллелограмм
из~задачи~\ref{geometry/linkage:problem:antiparallelogram},
точка~$O$ лежит на~прямой~$PQ$, которая параллельна $AD$, точка~$S$ такова,
что $OS = SP$, шарниры $O$ и~$S$ закреплены, стержни изображены сплошными
линиями.
Докажите, что при движении $P$ по~окружности точка~$Q$ движется по~прямой.

\end{problems}

Будем говорить, что шарнирный механизм \emph{рисует множество~$A$}, если
конфигурационное пространство некоторой вершины этого механизма совпадает
с~множеством~$A$.

\begin{problems}

\item
Постройте шарнирный механизм, который рисует со внутренностью
\\
\subproblem прямоугольник;
\quad
\subproblem треугольник.

\item\claim{Реверсор Кемпе}
\label{geometry/linkage:problem:kempe}%
Даны вещественные $a, b > 0$ и~целое $k \neq 0$.
Посмотрев на~рисунок~\ref{geometry/linkage:problem:kempe:fig},
постройте два шарнирных механизма, удовлетворяющих
следующему условию: в~них есть стержни $OA$, $OB$, $OC$, причем $|OA| = a$,
$|OB| = b$, и~если угол от~вектора~$OC$ к~вектору~$OA$ равен $\alpha$, то~угол
от~вектора~$OC$ к~вектору~$OB$ равен $k \alpha$ или $\alpha / k$:
\\
\subproblem для $k = 2$;
\quad
\subproblem для $k = -1$;
\\
\subproblem для произвольного натурального~$k$.

\begin{figure}[ht]
\begin{center}
    \jeolmfigure[width=0.4\textwidth]{kempe}
    \caption{к задаче~\ref{geometry/linkage:problem:kempe}.}
    \label{geometry/linkage:problem:kempe:fig}
\end{center}
\end{figure}

\item\claim{Сумматор Кемпе}
Даны вещественные $a, b, c > 0$.
Используя реверсор Кемпе, постройте шарнирный механизм, удовлетворяющий
следующему условию: в~нем есть стержни $OA$, $OB$, $OC$, $OD$, причем
$|OA| = a$, $|OB| = b$, $|OC| = c$ и~если угол от~вектора~$OD$ к~вектору~$OA$
равен $\alpha$, угол от~вектора~$OD$ к~вектору~$OB$ равен $\beta$, то~угол
от~вектора~$OD$ к~вектору~$OC$ равен $\alpha + \beta$.

\let\ov\overrightarrow

\item\claim{Транслятор Кемпе}
Даны вещественные $a, b > 0$.
Постройте шарнирный механизм, который содержит стержень~$OX$ и~шарниры
$Y$ и~$Z$, причем $|OX| = a$ и~в~каждой конфигурации при $|OY| \leq b$
выполняется $\ov{OZ} = \ov{OX} + \ov{OY}$.

\end{problems}

\claim{Теорема Кемпе}
Пусть $f(x, y)$~--- многочлен от~двух переменных, $D$~--- замкнутый круг
на~плоскости.
Тогда существует шарнирный механизм, который рисует множество
\[
    D \cap
    \bigl\{
        (x,y) \in \mathbb{R}
    \bigm|
        f(x, y) = 0
    \bigr\}
\; . \]

\begin{problems}

\item
Пусть радиус~$D$ равен $2 R$.
Рассмотрим ромб $OABC$, где $O$ совпадает с~началом координат, а~длины стержней
равны $R$.
Параметризуем точки круга~$D$ положением точки~$B$.
Пусть координаты точек $A$ и~$C$ равны
\(
    \bigl( R \cos(\alpha), R \sin(\alpha) \bigr)
\)
и~\(
    \bigl( R \cos(\beta), R \sin(\beta) \bigr)
\)
соответственно, тогда координаты точки
\[
    B
=
    (x, y)
=
    \bigl(
        R \cos(\alpha) + R \cos(\beta),
        R \sin(\alpha) + R \sin(\beta)
    \bigr)
\; . \]
\\
\subproblem
Докажите, что многочлен $f(x, y)$ может быть представлен в~виде
\[
    f
=
    \sum\limits_{|r| + |s| \leq n}
        f_{rs} \cdot \cos(r \alpha + s \beta + \gamma_{rs})
\; , \]
где $n$, $f_{rs}$, $\gamma_{rs}$~--- константы, зависящие от~многочлена~$f$,
а~$r$ и~$s$~--- целые числа.
\\
\subproblem
Докажите, что можно дополнить ромб так, чтобы в~полученном шарнирном механизме
существует вершина~$X$ такая, что для каждого положения $(x_0, y_0)$
вершины~$B$ абцисса вершины~$X$ одинакова и~равна $f(x_0, y_0)$.
\\
\subproblem
Соединив конструкцию из~предыдущего пункта и~инверсор Понселе, докажите
теорему Кемпе.

\end{problems}

\claim{Забавный факт}
Поскольку любую непрерывную кривую на~плоскости можно хорошо приблизить
многочленом от~двух переменных, то~теорему Кемпе можно интерпретировать
следующим образом: можно построить шарнирный механизм, который подделывает
вашу подпись.
Желающие могут проверить:
\\
\begingroup\providecommand\url{\texttt}%
\url{http://www.david.wf/linkage/}.
\endgroup

