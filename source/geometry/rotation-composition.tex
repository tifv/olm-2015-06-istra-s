% $date: 2015-06-27
% $timetable:
%   g10r2:
%     2015-06-27:
%       1:
%   g10r1:
%     2015-06-27:
%       2:

\section*{Повороты и клады}

% $authors:
% - Фёдор Бахарев

\begin{problems}

\item
На~сторонах треугольника $ABC$ построены во~внешнюю сторону квадраты
$A B B_1 A_2$ и $B C C_1 B_2$.
Точка~$X$~--- середина отрезка $B_1 B_2$.
Докажите, что
\\
\subproblem $2 BX = AC$;
\qquad
\subproblem $BX \perp AC$.

\item
Внутри выпуклого четырехугольника $ABCD$ построены равнобедренные прямоугольные
треугольники $A B O_1$, $B C O_2$, $C D O_3$ и~$D A O_4$.
Докажите, что если $O_1 = O_3$, то~$O_2 = O_4$.

\item
Диагональ~$AC$ является осью симметрии выпуклого четырехугольника $ABCD$.
Рассмотрим два правильных треугольника $ABK$ и~$BCM$ (точка~$K$ с~той~же
стороны от~$AB$, что и~точка~$C$, точка~$M$ расположена по~другую сторону
от~$BC$, чем точка~$A$).
Докажите, что точки $D$, $K$ и~$M$ лежат на~одной прямой.

\item
На~сторонах $AB$, $BC$ и~$CA$ треугольника $ABC$ взяты точки $P$, $Q$ и~$R$
соответственно.
Докажите, что центры описанных окружностей треугольников $APR$, $BPQ$ и~$CQR$
образуют треугольник, подобный треугольнику $ABC$.

\item
На~сторонах треугольника $ABC$ построены правильные треугольники $A'BC$
и~$B'AC$ внешним образом, $C'AB$~--- внутренним,
$M$~--- центр треугольника $C'AB$.
Докажите, что треугольник $A'B'M$ равнобедренный, причем
$\angle A'MB' = 120^\circ$.

\item
На~сторонах произвольного выпуклого четырехугольника внешним образом построены
квадраты.
Докажите, что отрезки, соединяющие центры противоположных квадратов, равны
по~длине и~перпендикулярны.

\item
На~сторонах выпуклого четырехугольника $ABCD$ во~внешнюю сторону построены
квадраты с~центрами $M$, $N$, $P$, $Q$.
Докажите, что середины диагоналей четырехугольников $ABCD$ и~$MNPQ$ образуют
квадрат.

\item
Две окружности пересекаются в~точках $A$ и~$B$.
Через точку~$A$ проведена прямая, которая повторно пересекает первую окружность
в~точке~$C$, а~вторую~--- в~точке~$D$
(точка~$A$ лежит между $C$ и~$D$).
Точки $K$ и~$L$~--- середины дуг $CB$ и~$DB$, не~содержащих точку~$A$,
а~$M$~--- середина отрезка~$CD$.
Докажите, что $\angle KML = 90^\circ$.

\item
Инструкция по~отысканию клада гласит:
\begin{quote}
Для того, чтобы найти клад, нужно стать под березой лицом к~прямой линии,
соединяющей дуб и~сосну.
При этом дуб должен оказаться справа, а~сосна слева.
Затем нужно пойти к~дубу, считая шаги.
Дойдя до~дуба, повернуть под прямым углом направо и~пройти столько~же шагов,
сколько было пройдено от~березы до~дуба.
В~этом месте остановиться и~поставить вешку.
Затем следует вернуться к~березе и~пойти от~нее к~сосне, считая шаги.
Дойдя до~сосны, повернуть под прямым углом налево и~пройти столько~же шагов,
сколько было пройдено от~березы до~сосны.
В~этом месте остановиться и~поставить вешку.
Клад зарыт точно посредине между вешками.
\end{quote}
Прибыв на~место, кладоискатель обнаружил, что дуб и~сосна налицо, а~от~березы
не~осталось и~следа.
Как найти клад за~наименьшее число попыток?

\item
Инструкция по~отысканию клада гласит:
\begin{quote}
Для того, чтобы найти клад, нужно встать под пальмой с~номером~<<1>>, лицом
к~пальме с~номером~<<2>>.
Затем нужно пройти ровно половину расстояния до~пальмы с~номером~<<2>>
и~повернуться лицом к~пальме с~номером~<<3>>.
Затем нужно пройти ровно треть расстояния до~пальмы с~номером~<<3>>
и~повернуться лицом к~пальме с~номером~<<4>>.
И~так далее.
На~$k$-ом этапе нужно пройти ровно $1 / (k + 1)$ расстояния до~пальмы
с~номером~<<$k+1$>> и~повернуться лицом к~пальме с~номером~<<$k+2$>>.
На~последнем этапе нужно пройти ровно $1 / 2015$ расстояния до~пальмы
с~номером~<<2015>>.
Если все сделано правильно,то~на~глубине двух метров точно под тобой будет
клад.
\end{quote}
Прибыв на~место, кладоискатель обнаружил, что на~всех 2015 пальмах стерлись
номера.
За~какое наименьшее число попыток он заведомо найдет клад?

\end{problems}

