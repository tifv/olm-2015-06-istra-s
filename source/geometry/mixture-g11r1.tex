% $date: 2015-06-25
% $timetable:
%   g11r1:
%     2015-06-25:
%       1:

% $caption: Разнобой (геометрия)

\section*{Разнобой}

% $authors:
% - Алексей Доледенок

\begin{problems}

\item
В~треугольнике $ABC$ проведена чевиана~$A A_1$.
Оказалось, что радиусы вписанных окружностей треугольников $A B A_1$
и~$A C A_1$ равны.
Докажите, что и~радиусы вневписанных окружностей, касающихся сторон
$B A_1$ и~$C A_1$, равны.

\item
Дан равносторонний треугольник $ABC$ и~прямая, проходящая через его центр.
Точки пересечения этой прямой со~сторонами $AB$ и~$AC$ отразили относительно
середин этих сторон.
Докажите, что прямая, проходящая через эти точки, касается вписанной окружности
треугольника $ABC$.

\item
Для остроугольного треугольника $ABC$ и~произвольной точки~$X$ внутри него
обозначим через $A_X$, $B_X$, $C_X$ точки пересечения прямых $AX$, $BX$, $CX$
со~сторонами $BC$, $AC$, $AB$ соответственно.
Сколько точек внутри треугольника $ABC$ могут обладать тем свойством, что
$\angle A A_X C = \angle B B_X A = \angle C C_X B$?

\item
\subproblem\emph{Окружность Конвея.}
В~произвольном треугольнике $ABC$ на~прямых $AB$ и~$AC$ отложим от~точки~$A$
вовне треугольника отрезки, равные $BC$.
Концы этих отрезков обозначим $A_1$ и~$A_2$.
Аналогично определим точки $B_1$, $B_2$, $C_1$, $C_2$.
Докажите, что эти шесть точек лежат на~одной окружности.
\\
\subproblem\emph{Окружность Тэйлора.}
В~остроугольном треугольнике $ABC$ опустим перпендикуляры на~стороны
треугольника из~оснований высот.
Докажите, что шесть получившихся точек лежат на~одной окружности, причем она
является окружностью Конвея для треугольника с~вершинами в~серединах сторон
ортотреугольника.

\item
Внутри треугольника $ABC$ выбрана произвольная точка~$M$.
Докажите, что
\(
    MA + MB + MC
\leq
    \max (AB + BC, BC + CA, CA + AB)
\).

\item
В~треугольнике $ABC$ на~биссектрисе~$A A_1$ выбрана точка~$O$ так, что
$\angle OBC = \angle A + \angle C$.
Пусть $B_1$ и~$C_1$~--- точки пересечения прямых $BO$ и~$CO$ со~сторонами $AC$
и~$AB$ соответственно.
Докажите, что $\angle A_1 B_1 C_1 = 90^{\circ}$.

\item
В~некоторой трапеции сумма боковой стороны и~диагонали равна сумме другой
боковой стороны и~другой диагонали.
Докажите, что трапеция равнобокая.

\end{problems}

