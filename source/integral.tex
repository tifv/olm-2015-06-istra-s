% $groups$delegate: false
% $groups$delegate$into: false
% $groups$matter: false
% $groups$matter$into: false

% $matter[-header,-no-header]:
% - .[no-header]

% $matter[-matter-guard,no-header]:
% - verbatim: \begingroup \let\ifsourcelinks\iftrue
%   condition: source-link
% - .[matter-guard]
% - verbatim: \endgroup % \let\ifsourcelinks
%   condition: source-link

\begingroup
\providecommand\ifsourcelinks{\iffalse}
\providecommand\url{\texttt}

\strut

\vfill

\vfill

\strut

\clearpage


\subsection*{Немного о~группах}

Занятия проходили в шести группах:
\begingroup\multicolsep=\parskip
\begin{multicols}{3}
9    <<Тираннозавры>>
\\
10-2 <<Трицератопсы>>
\\
10-1 <<Диплодоки>>
\\
11-2 <<Археоптериксы>>
\\
11-1 <<Птеродактили>>
\\
X    <<Аллозавры>>
\end{multicols}
\endgroup


\subsection*{Немного о~структуре}

Материалы разбиты по~группам, в~пределах каждой группы отсортированы по~темам:
\begin{itemize}
    \item \emph{тренировочные олимпиады;}
    \item алгебра;
    \begin{itemize}
        \item теория чисел;
        \item многочлены;
        \item неравенства;
    \end{itemize}
    \item геометрия;
    \item комбинаторика;
    \begin{itemize}
        \item теория графов.
    \end{itemize}
\end{itemize}

Материалы, общие для нескольких групп, дублируются.
\ifsourcelinks
Все материалы сопровождаются ссылками на~исходные файлы \LaTeX.
\fi


\subsection*{Немного об~авторах}


\endgroup

