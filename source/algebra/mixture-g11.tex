% $date: 2015-06-15
% $timetable:
%   g11r1:
%     2015-06-15:
%       2:
%   g11r2:
%     2015-06-15:
%       1:

\section*{Алгебраический разнобой}

% $authors:
% - Андрей Меньщиков

% $build$style[print,master]:
% - .[resize-to]

\begin{problems}

\item
Решите уравнение
\(
    \max \{ x^2 + y^2, 2 \} = \min \{ -2 x, 2 y \}
\).

\item
Действительные числа $x$, $y$, $z$ принадлежат отрезку $[0; 1]$.
Докажите, что
\[
    x^2 + y^2 + z^2 \leq x^2 y + y^2 z + z^2 x + 1
\,.\]

\item
Число $(1 + \sqrt{2})^{2015}$ записали в~виде $a + b\sqrt{2}$ с~целыми $a$
и~$b$.
Докажите, что $|a^2 - 2 b^2| = 1$.

\item
Найдите все функции $f \colon \mathbb{R} \to \mathbb{R}$ такие, что для всех
$x, y \in \mathbb{R}$ справедливо
\[
    f(x^2 + y^2 + 2 x y) = f(x^2) + f(y^2) + 2 x y
\,.\]

\item
Действительные числа $a, b, c, d$, по~модулю большие единицы, удовлетворяют
соотношению
\(
    a b c + a b d + a c d + b c d + a + b + c + d = 0
\).
Докажите, что
\[
    \frac{1}{a - 1} + \frac{1}{b - 1} + \frac{1}{c - 1} + \frac{1}{d - 1}
>
    0
\,.\]

\item
Про действительные числа $a$, $b$ и~$c$ известно, что сумма дробей
\[
    \frac{a^2 + b^2 - c^2}{2 a b}
\;,\enspace
    \frac{b^2 + c^2 - a^2}{2 b c}
\text{\enspace и\enspace}
    \frac{a^2 + c^2 - b^2}{2 a c}
\]
равна единице.
Докажите, что из~трех данных дробей две равны $1$, а~одна равна $-1$.

\item
Существует~ли многочлен от~$x$ и~$y$, множество значений которого есть
в~точности множество положительных чисел?

\item
Положительные иррациональные числа $p$ и~$q$ таковы, что
$1 / p + 1 / q = 1$.
Докажите, что каждое натуральное число является членом ровно одной
из~последовательностей $[n p]$ и~$[n q]$.

\item
Найдите все функции $f \colon \mathbb{R} \to \mathbb{R}$ такие, что для всех
$x,y \in \mathbb{R}$ справедливо
\[
    f(x) \leq x
\text{ и }
    f(x + y) \leq f(x) + f(y)
\,.\]

\item
Дано натуральное число $N > 3$.
Назовем набор из~$N$ точек на~координатной плоскости \emph{допустимым}, если
их~абсциссы различны, и~каждая из~этих точек окрашена либо в~красный, либо
в~синий цвет.
Будем говорить, что многочлен $P(x)$ \emph{разделяет} допустимый набор точек,
если либо выше графика $P(x)$ нет красных точек, а~ниже --- нет синих, либо
наоборот (на~самом графике могут лежать точки обоих цветов).
При каком наименьшем $k$ любой допустимый набор из~$N$ точек можно разделить
многочленом степени не~более $k$?

\end{problems}

