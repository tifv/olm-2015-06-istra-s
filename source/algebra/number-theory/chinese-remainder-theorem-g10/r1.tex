% $date: 2015-06-23
% $timetable:
%   g10r1:
%     2015-06-23:
%       1:

\section*{Китайская теорема об остатках}

% $authors:
% - Владимир Алексеевич Брагин
% - Иван Викторович Митрофанов

Пусть $m_{1}$, $m_{2}$, \ldots, $m_{n}$~--- попарно взаимно простые натуральные
числа.
\\
\textbf{Китайская теорема об~остатках} утверждает, что для любых целых чисел
$a_{1}$, $a_{2}$, \ldots, $a_{n}$ существует такое натуральное~$x$, что
$x \equiv a_i  \pmod{m_{i}}$.
Более того, такое $x$ единственное с~точностью до~добавления кратного
$M = m_{1} m_{2} \ldots m_{n}$.

\begin{problems}

\item
Докажите Китайскую теорему об~остатках.

\item
Генерал построил солдат в~колонну по~4, но~при этом солдат Иванов остался
лишним.
Тогда генерал построил солдат в~колонну по~5.
И~снова Иванов остался лишним.
Когда~же и~в~колонне по~6 Иванов оказался лишним, генерал посулил ему наряд вне
очереди, после чего в~колонне по~7 Иванов нашел себе место и~никого лишнего
не~осталось.
Сколько солдат могло быть у~генерала?

\item
Многочлен с~целыми коэффициентами при некоторых целых значениях аргумента
делится на~1001, а~при некоторых целых значениях аргумента делится на~1000.
Докажите, что при некоторых значениях аргумента значение многочлена делится
на~$1\,001\,000$.

\item
Докажите, что существуют 18 последовательных натуральных чисел, среди которых
нет числа, взаимно простого с~остальными.

\item
Существует~ли такое 2000-значное составное число, которое при замене любой
тройки соседних цифр на~произвольную тройку цифр остается составным?

\item
Дана бесконечная арифметическая прогрессия из~натуральных чисел.
Докажите, что для любого натурального~$N$ в~этой прогрессии можно найти $N$
составных чисел, идущих подряд.

\item
Для каких натуральных $n > 1$ существуют натуральные
$b_1$, $b_2$, \ldots, $b_n$ (не~все из~которых равны) такие, что для любого
натурального~$k$ число $(b_{1} + k) (b_{2} + k) \ldots (b_{n} + k)$ является
точной степенью?
(Показатель может зависеть от~$k$ но~всегда быть больше $1$.)

\item
Число называется \emph{плохим,} если не~делится на~2, 3, 5, 7, 11, 13, 17.
Число называется \emph{хорошим,} если делится на~2 или больше из~этих чисел.
Какое наибольшее количество плохих чисел можно выбрать так, чтобы сумма любых
двух их них была хорошим числом?

\item
Пусть $n$~--- натуральное число, которого ровно $k$ различых простых делителей.
\\
\subproblem
Сколько существует попарно несравнимых по~модулю~$n$ таких целых~$a$, что
$(a^2 - a)$ делится на~$n$?
\\
\subproblem
Докажите, что существует натуральное $a$, $1 < a < n / k + 10$, такое что
$(a^2 - a)$ делится на~$n$.

\end{problems}

