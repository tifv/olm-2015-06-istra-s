% $date: 2015-06-25
% $timetable:
%   g11r2:
%     2015-06-25:
%       1:

% $caption: Разнобой по теории чисел

\section*{Разнобой по ТЧ}

% $authors:
% - Лев Шабанов

\begin{problems}

\item
Существуют~ли четыре последовательных натуральных числа, каждое из~которых
можно представить в~виде суммы квадратов двух натуральных чисел?

\item
Пусть $m$, $n$~--- взаимно простые натуральные числа.
Найдите наибольшее возможное значение
$\operatorname{\text{НОД}}(9 m + 7 n, 3 m + 2 n)$.

\item
Найти такие 2015 натуральных чисел, что ни~одно из~них не~делится на~другое,
а~произведение любых двух из~них делится на~любое из~оставшихся чисел.

\item
Докажите, что:
\\[0.5ex]
\subproblem
$\underbrace{11 \ldots 11}_{12}$ делится на~13;
\qquad
\subproblem
$\underbrace{11 \ldots 11}_{288}$ делится на~323;
\\[0.5ex]
\subproblem
$\underbrace{11 \ldots 11}_{420}$ делится на~539.

\item
Докажите, что сумма квадратов пяти последовательных целых чисел не~может быть
полным квадратом.

\item
Камни лежат в~трех кучках: в~одной~--- 51~камень, в~другой~--- 49~камней,
а~в~третьей~--- 5~камней.
Разрешается объединять любые кучки в~одну, а~также разделять кучку из~четного
количества камней на~две равные.
Можно~ли получить 105 кучек по~одному камню в~каждой?

\item
Найдите наименьшее натуральное число, половина которого~--- седьмая степень,
треть~--- пятая степень, а~четверть~--- тринадцатая степень.

\item
По~кругу расставлены 99 натуральных чисел.
Известно, что каждые два соседних числа отличаются или на~1, или на~2,
или в~два раза.
Докажите, что хотя~бы одно из~этих чисел делится на~3.

\item
Пусть $P(x)$~--- многочлен с~целыми коэффициентами.
Оказалось, что $P(8)$, $P(12)$ и~$P(14)$ делятся на~1001.
Докажите, что тогда сумма коэффициентов $P(x)$ также делится на~1001.

\item
Может~ли наименьшее общее кратное целых чисел $1, 2, \ldots, n$ быть
в~2016 раз больше, чем наименьшее общее кратное целых чисел $1, 2, \ldots, m$?

\item
Докажите, что существует натуральное число, которое при замене любой тройки
соседних цифр на~произвольную тройку остается составным.

\end{problems}

