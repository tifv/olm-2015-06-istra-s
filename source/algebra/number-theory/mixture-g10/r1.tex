% $date: 2015-06-16
% $timetable:
%   g10r1:
%     2015-06-16:
%       1:

\section*{Остатки и показатели}

% $authors:
% - Владимир Брагин

\begin{problems}

\item
Существуют~ли такие натуральные $a$ и~$b$, что $a^2 + 2 b^2 = 1001$?

\item
Докажите, что $a^5 + b^5 + c^5 + 5$ не~может быть точной пятой степенью.

\item
Решите в~натуральных числах уравнение $7^m - 2^n = 3$.

\item
Решите в~натуральных числах уравнение $5^x = 2^x + 1$.

\item
Решите в~натуральных числах уравнение $7^x = 2^y 3^z + 1$.

\end{problems}

\claim{Напоминание}
Пусть $a$ и~$n$~--- взаимно простые натуральные числа.
Тогда \emph{показателем $a$ по~модулю $d$} называется наименьшее
натуральное~$d$ такое, что $a^d \equiv 1 \pmod{n}$.

\begin{problems}

\item
\sp
Докажите, что любой нечетный простой делитель числа $a^{2^k} + 1$ имеет вид
$2^{k+1} x + 1$.
\\
\sp
Докажите, что если $a^4 + b^4$ делится на~101, то~$a$ и~$b$ делятся на~101.
\\
\sp
Числа $(16 a + 15 b)$ и~$(15 a - 16 b)$ являются полными квадратами.
Какое наименьшее значение может принимать число $(15 a - 16 b)$?

\item
Пусть $p$, $q$~--- простые числа, большие 5.
Оказалось, что $2^p + 3^p$ делятся на~$q$.
Докажите, что $(q - 1)$ делится на~$2 p$.

\item
Докажите, что $(2^n - 1)$ не~делится на~$n$, если $n > 1$.

\item
Какие остатки могут принимать степени двойки при делении на~$3^{2015}$?

\end{problems}

