% $date: 2015-06-16
% $timetable:
%   g11r1:
%     2015-06-16:
%       2:

\section*{Числовой разнобой}

% $authors:
% - Андрей Меньщиков

\begin{problems}

\item
При каких целых $k$ число $(a^3 + b^3 + c^3 - k a b c)$ делится на~$a + b + c$
при любых целых $a$, $b$, $c$, сумма которых не~равна 0?

\item
Натуральные числа $a$, $x$ и~$y$, большие 100, таковы, что
$(y^2 - 1) = a^2 (x^2 - 1)$.
Какое наименьшее значение может принимать дробь $a / x$?

\item
Пусть $n > 1$~--- натуральное число.
Выпишем дроби
\[
    \frac{1}{n}, \frac{2}{n}, \ldots, \frac{n - 1}{n}
\]
и~приведем каждую из~них к~несократимому виду;
сумму числителей полученных дробей обозначим через $f(n)$.
При каких натуральных $n > 1$ числа $f(n)$ и~$f(2015 n)$ имеют разную четность?

\item
Последовательность $\{ a_n \}$ задана рекуррентной формулой и~начальными
значениями:
\[
    a_1 = 257
,\enspace
    a_2 = 2015
,\quad
    a_{n+2} = a_{n+1}^2 + a_n^2
.\]
Докажите, что ни~один из~членов последовательности не~делится на~2011.

\item
Пусть $d_1 < d_2 < \ldots < d_k$~--- все натуральные делители числа~$n$.
Докажите, что
\[
    d_1 d_2 + d_2 d_3 + \ldots + d_{k-1} d_k < n^2
.\]

\item
Найдите наименьшее простое~$p$ такое, что $(2^{120!} - 1)$ делится на~$p$,
но~не~делится на~$p^2$.

\item
Последовательность $x_1, x_2, \ldots$ задана правилами: $x_1 = 2$,
$x_{n+1}$~--- наибольший простой делитель числа
\(
    x_1 \cdot x_2 \cdot \ldots \cdot x_{n} + 1
\)
при всех $n \geq 1$.
Докажите, что ни~один из~членов последовательности не~равен 5.

\item
Натуральные числа $x$, $y$, $m$ и~$n$ таковы, что
$n^2 = 15 x + 16 y$ и~$m^2 = 16 x - 15 y$.
Какое наименьшее значение может принимать величина $\min (m, n)$?

\item
Три натуральных числа таковы, что сумма их~попарных произведений составляет
половину точного квадрата, а~сумма их~квадратов является степенью простого
числа.
Какие значения может принимать это простое число?

\item
Пусть $p$~--- простое число.
Про целые числа $x_1, x_2, \ldots, x_p$ известно, что
$x_1^n + x_2^n + \ldots + x_p^n$ делится на~$p$ при любом натуральном~$n$.
Докажите, что $(x_1 - x_2)$ делится на~$p$.

\end{problems}

