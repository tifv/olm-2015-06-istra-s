% $date: 2015-06-18
% $timetable:
%   g11r2:
%     2015-06-18:
%       1:

\section*{Многочлены}

% $authors:
% - Андрей Меньщиков

\begin{problems}

\item
$P(x)$ и~$Q(x)$ -- приведённые многочлены десятой степени, графики которых
не~пересекаются.
Докажите, что уравнение $P(x + 1) = Q(x - 1)$ имеет хотя~бы один действительный
корень.

\item
Многочлен с~целыми коэффициентами в~четырех целых точках принимает
значение~$-2$.
Может~ли он в~какой-нибудь целой точке принимать значение $2015$?

\item
Многочлен $Q(x)$ таков, что уравнение $Q(x) = x$ не~имеет действительных
корней.
Докажите, что уравнение $Q(Q(x)) = x$ также не~имеет действительных корней.

\item
Даны два многочлена положительной степени $P(x)$ и~$Q(x)$, причем выполнены
тождества $P(P(x)) = Q(Q(x))$ и~$P(P(P(x))) = Q(Q(Q(x)))$.
Обязательно~ли тогда выполнено тождество $P(x) = Q(x)$?

\item
Многочлен с целыми коэффициентами $P(x)$ и~бесконечная последовательность
различных целых чисел $a_1, a_2, \ldots$ таковы, что
$P(a_1) = 0$, $P(a_2) = a_1$, $P(a_3) = a_2$ и~т.~д.
Какую степень и~старший коэффициент может иметь $P(x)$?

\item
Все коэффициенты многочлена $P(x)$~--- целые числа.
Известно, что уравнения $P(x) = 1$, $P(x) = 2$, $P(x) = 3$ имеют целые корни.
Докажите, что уравнение $P(x) = 5$ не~может иметь двух или более целых корней.

\item
Докажите, что функция $f(n) = 1^k + 2^k + \ldots + n^k$ является многочленом
$(k + 1)$-ой степени от~переменной~$n$ с~рациональными коэффициентами.

\item
Существует~ли многочлен от~$x$ и~$y$, принимающий положительные значения в~тех
и~только в~тех точках, обе координаты которых положительны?

%\item
%Шесть членов команды Судьбы на~Международной математической олимпиаде
%отбираются из~13 кандидатов.
%На~отборочной олимпиаде кандидаты набрали $a_1, a_2, \ldots, a_{13}$ баллов
%($a_i \neq a_j$ при $i \neq j$).
%Руководитель команды заранее выбрал 6 кандидатов и~теперь хочет, чтобы
%в~команду попали именно они.
%С~этой целью он подбирает многочлен $P(x)$ и~вычисляет \emph{творческий
%потенциал} каждого кандадата по~формуле $c_i = P(a_i)$.
%При каком наименьшем $n$ он заведомо сможет подобрать такой многочлен $P(x)$
%степени не~выше $n$, что творческий потенциал любого из~его шести кандидатов
%окажется строго больше, чем у~каждого из~семи оставшегося?

\end{problems}

