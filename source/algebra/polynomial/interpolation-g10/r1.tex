% $date: 2015-06-24
% $timetable:
%   g10r1:
%     2015-06-24:
%       2:

\section*{Китайская теорема об остатках для многочленов. Интерполяция}

% $authors:
% - Владимир Алексеевич Брагин
% - Иван Викторович Митрофанов

\begin{problems}

\item \emph{(воспоминания о~вчерашнем)}
Найдите все такие целые~$n$, что
$n \equiv 17 \pmod{54}$, $n \equiv 15 \pmod{80}$ и~$n \equiv 65 \pmod{75}$.

\item
Многочлен $P(x)$ дает остаток $2 x + 1$ при делении на~при делении
на~$x^2 + x + 1$ и~остаток $3 x - 1$ при делении на~$x^2 - x + 1$.
Какой остаток дает многочлен $P(x)$ при делении на~$x^4 + x^2 + 1$?

\item
Пусть многочлены $P(x)$ и~$Q(x)$ взаимно простые.
Для произвольных многочленов $h(x)$ и~$g(x)$ докажите, что существует такой
многочлен $f(x)$, что $f(x) - h(x)$ делится на~$P(x)$, а~$f(x) - g(x)$ делится
на~$Q(x)$.

\end{problems}

Пусть даны два набора чисел:
$x_{0}$, $x_{1}$, \ldots, $x_{n}$ и~$y_{0}, y_{1}, \ldots, y_{n}$, причем
в~первом наборе все числа различны.
Тогда требуется найти многочлен~$F$ степени не~выше $k$ такой, что
$F(x_{i}) = y_{i}$ при $i = 0, 1, \ldots, n$.

\begin{problems}

\item
\subproblem
Докажите, что при $k = n$ решения либо нет вовсе, либо оно единственное.
\\
\subproblem
Докажите, что при $k > n$ решения либо нет вовсе, либо их бесконечно много.

\end{problems}

Будем доказывать, что решение существует.

\begin{problems}

\item
\subproblem
Пусть многочлен $g_{k}(x)$ степени~$n$ равен $0$ при всех $x_{i}$
($i = 0, 1, \ldots, n$), кроме $x_{k}$.
Докажите, что
\[
    g_{k}(x)
=
    c \cdot \prod_{\substack{0 \leq i \leq n \\ i \neq k}}
        (x - x_{i})
\, . \]
\\
\subproblem
Чему должна быть равна константа~$c$, чтобы $g_{k}(x_{k})$ было равно $1$?
\\
\subproblem
\claim{Интерполяционный многочлен Лагранжа}
Докажите, что искомый многочлен~$f(x)$ равен
\[
    \sum_{k=0}^{n}
        \left(
            y_{k} \cdot
            \prod_{\substack{0 \leq i \leq n \\ i \neq k}}
                \dfrac{x - x_{i}}{x_{k} - x_{i}}
        \right)
=
    \sum_{k=0}^{n}
        \bigl( y_{k} \cdot g_{k}(x) \bigr)
\; . \]

\item
Многочлен $f(x)$ степени~$n$ принимает в~целых точках целые значения.
Докажите, что
\\
\subproblem
все его коэффициенты рациональны;
\\
\subproblem
знаменатели всех коэффициентов делят $n!$.

\item
Для любых различных $a$, $b$, $c$, $d$, $e$ докажите, что
\begin{align*} &
    \frac{(a - b) (a - c) (a - d)}{(e - b) (e - c) (e - d)} +
    \frac{(a - b) (a - c) (a - e)}{(d - b) (d - c) (d - e)}
    + \\ & +
    \frac{(a - b) (a - d) (a - e)}{(c - b) (c - d) (c - e)} +
    \frac{(a - c) (a - d) (a - e)}{(b - c) (b - d) (b - e)}
=
    1
\, . \end{align*}

\item
Многочлены $P(x)$ и~$Q(x)$ удовлетворяют тождеству
$x^{2013} = (x^{3} - 6 x + 11 x - 6) \cdot P(x) + Q(x)$, причем степень
многочлена $Q(x)$ меньше $3$.
\\
\subproblem
Найдите многочлен $Q(x)$.
\\
\subproblem
\emph{(немного в~сторону, но~тоже важно)}
Докажите, что у~многочлена $P(x)$  все коэффициенты положительные.

\item
Функция $f(x)$ при целых $x$ принимает целые значения.
Оказалось, что для любого простого~$p$ существует такой многочлен $Q_{p}(x)$
с~целыми коэффицентами степени не~выше $2013$, что $f(n) - Q_{p}(n)$ делится
на~$p$ при любом целом~$n$.
Докажите, что существует такой многочлен~$g(x)$ с~рациональными коэффициентами,
что $f(n) = g(n)$ при любом натуральном~$n$.

\end{problems}

